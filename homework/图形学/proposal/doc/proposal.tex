%%%%% --------------------------------------------------------------------------------
%%
%%%%********************************* Proposal ***************************************
%%
%%% ++++++++++++++++++++++++++++++++++++++++++++++++++++++++++++++++++++++++++++++++++
\part{\FakeBold{开题报告}} % (fold)
\label{prt:开题报告_}
	\setcounter{section}{0}
	\section{研究开发的背景、意义与目的} % (fold)
	\label{sec:研究开发的背景_意义与目的}
		开题报告是在完成文献调研后撰写的关于论文选题与如何实施的论述性报告,目的是要请老师及专家们帮忙判断一下所研究的选题有没有价值、研究方法及技术路线是否可行、预期目标及进度安排是否合理等。因此开题报告应重点阐述研究的主要内容、拟解决的主要问题、拟采用的研究方法等。

		此部分内容的“背景”应与毕业设计的内容紧密相关,“意义与目的”可以论述前人工作或现有方法等的不足、说明毕业设计所要进行的研究开发工作的意义或价值等。

		以下二级标题仅为说明格式要求。

		\subsection{背景介绍} % (fold)
		\label{sub:背景介绍}
		
		% subsection 背景介绍 (end)
		\subsection{本研究的意义和目的} % (fold)
		\label{sub:本研究的意义和目的}
		
		% subsection 本研究的意义和目的 (end)
	% section 研究开发的背景_意义与目的 (end)
	\newpage
	\section{主要研究开发内容} % (fold)
	\label{sec:主要研究开发内容}
		\subsection{主要研究内容} % (fold)
		\label{sub:主要研究内容}
		
		% subsection 主要研究内容 (end)
		\subsection{技术路线} % (fold)
		\label{sub:技术路线}
		
		% subsection 技术路线 (end)
		\subsection{可行性分析} % (fold)
		\label{sub:可行性分析}
		
		% subsection 可行性分析 (end)
	% section 主要研究开发内容 (end)
	\newpage
	\section{进度安排及预期目标} % (fold)
	\label{sec:进度安排及预期目标}
		\subsection{进度安排} % (fold)
		\label{sub:进度安排}
		
		% subsection 进度安排 (end)
		\subsection{预期目标} % (fold)
		\label{sub:预期目标}
		
		% subsection 预期目标 (end)
	% section 进度安排及预期目标 (end)
	\newpage
	\section*{参考文献} % (fold)
	\label{sec:参考文献2}
	\addcontentsline{toc}{section}{\protect\numberline{}参考文献}%
	
	% section* 参考文献 (end)
% part 开题报告_ (end)

%%% ++++++++++++++++++++++++++++++++++++++++++++++++++++++++++++++++++++++++++++++++++
