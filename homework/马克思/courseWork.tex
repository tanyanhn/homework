\documentclass[a4paper]{book}

\usepackage{geometry}
% make full use of A4 papers
\geometry{margin=1.5cm, vmargin={0pt,1cm}}
\setlength{\topmargin}{-1cm}
\setlength{\paperheight}{29.7cm}
\setlength{\textheight}{25.1cm}

% auto adjust the marginals
\usepackage{marginfix}

\usepackage{amsfonts}
\usepackage{amsmath}
\usepackage{amssymb}
\usepackage{amsthm}
%\usepackage{CJKutf8}   % for Chinese characters
\usepackage{ctex}
\usepackage{enumerate}
\usepackage{graphicx}  % for figures
\usepackage{layout}
\usepackage{multicol}  % multiple columns to reduce number of pages
\usepackage{mathrsfs}  
\usepackage{fancyhdr}
\usepackage{subfigure}
\usepackage{tcolorbox}
\usepackage{tikz-cd}
\usepackage{listings}
\usepackage{xcolor} %代码高亮
\usepackage{braket}
\usepackage{algorithm} 
\usepackage{algorithmicx}  
\usepackage{algpseudocode}  
\usepackage{amsmath}  
\usepackage[fontsize=13pt]{fontsize}

\floatname{algorithm}{算法}  
\renewcommand{\algorithmicrequire}{\textbf{输入:}}  
\renewcommand{\algorithmicensure}{\textbf{输出:}}  
\renewcommand{\algorithmicrequire}{\textbf{Input : }}
\renewcommand{\algorithmicrequire}{\textbf{Precondition : }}
\renewcommand{\algorithmicensure}{\textbf{Output : }}
\renewcommand{\algorithmicensure}{\textbf{Postcondition : }}
%------------------
% common commands %
%------------------
% differentiation
\newcommand{\gen}[1]{\left\langle #1 \right\rangle}
\newcommand{\dif}{\mathrm{d}}
\newcommand{\difPx}[1]{\frac{\partial #1}{\partial x}}
\newcommand{\difPy}[1]{\frac{\partial #1}{\partial y}}
\newcommand{\Dim}{\mathrm{D}}
\newcommand{\avg}[1]{\left\langle #1 \right\rangle}
\newcommand{\sgn}{\mathrm{sgn}}
\newcommand{\Span}{\mathrm{span}}
\newcommand{\dom}{\mathrm{dom}}
\newcommand{\Arity}{\mathrm{arity}}
\newcommand{\Int}{\mathrm{Int}}
\newcommand{\Ext}{\mathrm{Ext}}
\newcommand{\Cl}{\mathrm{Cl}}
\newcommand{\Fr}{\mathrm{Fr}}
% group is generated by
\newcommand{\grb}[1]{\left\langle #1 \right\rangle}
% rank
\newcommand{\rank}{\mathrm{rank}}
\newcommand{\Iden}{\mathrm{Id}}

% this environment is for solutions of examples and exercises
\newenvironment{solution}%
{\noindent\textbf{Solution.}}%
{\qedhere}
% the following command is for disabling environments
%  so that their contents do not show up in the pdf.
\makeatletter
\newcommand{\voidenvironment}[1]{%
\expandafter\providecommand\csname env@#1@save@env\endcsname{}%
\expandafter\providecommand\csname env@#1@process\endcsname{}%
\@ifundefined{#1}{}{\RenewEnviron{#1}{}}%
}
\makeatother

%---------------------------------------------
% commands specifically for complex analysis %
%---------------------------------------------
% complex conjugate
\newcommand{\ccg}[1]{\overline{#1}}
% the imaginary unit
\newcommand{\ii}{\mathbf{i}}
%\newcommand{\ii}{\boldsymbol{i}}
% the real part
\newcommand{\Rez}{\mathrm{Re}\,}
% the imaginary part
\newcommand{\Imz}{\mathrm{Im}\,}
% punctured complex plane
\newcommand{\pcp}{\mathbb{C}^{\bullet}}
% the principle branch of the logarithm
\newcommand{\Log}{\mathrm{Log}}
% the principle value of a nonzero complex number
\newcommand{\Arg}{\mathrm{Arg}}
\newcommand{\Null}{\mathrm{null}}
\newcommand{\Range}{\mathrm{range}}
\newcommand{\Ker}{\mathrm{ker}}
\newcommand{\Iso}{\mathrm{Iso}}
\newcommand{\Aut}{\mathrm{Aut}}
\newcommand{\ord}{\mathrm{ord}}
\newcommand{\Res}{\mathrm{Res}}
%\newcommand{\GL2R}{\mathrm{GL}(2,\mathbb{R})}
\newcommand{\GL}{\mathrm{GL}}
\newcommand{\SL}{\mathrm{SL}}
\newcommand{\Dist}[2]{\left|{#1}-{#2}\right|}

\newcommand\tbbint{{-\mkern -16mu\int}}
\newcommand\tbint{{\mathchar '26\mkern -14mu\int}}
\newcommand\dbbint{{-\mkern -19mu\int}}
\newcommand\dbint{{\mathchar '26\mkern -18mu\int}}
\newcommand\bint{
{\mathchoice{\dbint}{\tbint}{\tbint}{\tbint}}
}
\newcommand\bbint{
{\mathchoice{\dbbint}{\tbbint}{\tbbint}{\tbbint}}
}





%----------------------------------------
% theorem and theorem-like environments %
%----------------------------------------
\numberwithin{equation}{chapter}
\theoremstyle{definition}

\newtheorem{thm}{Theorem}[chapter]
\newtheorem{axm}[thm]{Axiom}
\newtheorem{alg}[thm]{Algorithm}
\newtheorem{asm}[thm]{Assumption}
\newtheorem{defn}[thm]{Definition}
\newtheorem{prop}[thm]{Proposition}
\newtheorem{rul}[thm]{Rule}
\newtheorem{coro}[thm]{Corollary}
\newtheorem{lem}[thm]{Lemma}
\newtheorem{exm}{Example}[chapter]
\newtheorem{rem}{Remark}[chapter]
\newtheorem{exc}[exm]{Exercise}
\newtheorem{frm}[thm]{Formula}
\newtheorem{ntn}{Notation}

% for complying with the convention in the textbook
\newtheorem{rmk}[thm]{Remark}


%----------------------
% the end of preamble %
%----------------------

\begin{document}
\pagestyle{plain}
\pagenumbering{roman}
\pagenumbering{arabic}
\setcounter{page}{1}
\section*{\centering \huge 双减政策的思考}

2021 年 7 月 24 日,
中共中央办公厅、
国务院
办公厅印发《关于进一步减轻义务教育阶段学生
作业负担和校外培训负担的意见》
(以下简称《意
见》),
明确提出全面压减作业总量和作业时长,
提升课后服务水平,
全面规范校外培训行为,
大
力提升教育教学质量
。
《意见》提出的减少义务
教育阶段学生作业负担和校外培训负担相关政
策(以下简称“双减”政策),
不仅对义务教育阶段
学校和教师的教育质量提出了更高要求,
也对各
地政府、教培行业、学生及家长等利益相关者产
生直接或间接的影响。与其他公共政策相比,
教育政策的利益主体更加庞杂 ,
所构成的各利益
主体间的利益博弈关系和形式也更加交错复
杂 。

“双减”政策首批试点的城市有北京、上海、
广州、郑州等 9 座城市。各地为落实“双减”政
策,
纷纷出台相应举措。例如 ,
北京发布《关于
进一步减轻义务教育阶段学生作业负担和校外
培训负担的措施》,
上海发布《关于进一步减轻
义务教育阶段学生作业负担和校外培训负担的
实施意见》。社会各界对“双减”政策的根本目的展开
了诸多解读和讨论。第一 ,
“双减”政策的根本
目的是拨乱反正 ,
推动义务教育公共服务体系
回归公共性\cite{ref1} ,
全面恢复教育生态。近十多年
来,
校外培训机构、私人家教盛行 ,
培训机构之
间的竞争、运作、价格等机制交叠影响 ,
对学校
教育产生了巨大了冲击和干扰 ,
导致区域间教
育发展不平衡现象加剧、
“学校教育做减法、课
外培训做加法”、教育体系内角色定位混乱等问
题。
“双减”政策通过规范校外学科培训市场,
能
够有力地打击资本逐利的市场乱象 ,
重新定位
学校教育与校外教育之间的关系 ,
促进整个社
会的教育体系重建 ,
回归教育的公益性、共享
性\cite{ref2}。第二 ,
“双减”政策旨在强调学校教育的主
阵地作用,
落实学校教育立德树人的根本任务。
在资本市场制造的教育焦虑环境下 ,
大量学生
和家长基于“理性经济人”的立场选择“调动家
庭经济文化资本参与内卷化教育竞争” 、加入
市场化校外培训机构 ,
课后“跑班”现象极为普
遍。
“双减”政策重拳出击,
是对校外教育无序竞
争乱象的综合治理 ,
能够破除“校内上课、校外
补课”的双轨困境 ,增强学校教育的主阵地作
用,
进而通过学校教育提质增效实现立德树人
目标。


对学校来说,“双减”政策对学校应当承担的责
任进行了明确规定,
包括着力提高教学质量、作
业管理水平和课后服务水平等 ,
对学校有了更
多内容、更高标准的工作要求。学校要承担的
工作责任更多,
面临更大的工作压力。同时,
社
会群体普遍对学校提供的教育质量和服务水平
缺乏信心 ,
更加剧学校教育的压力。如何提高
教学质量、安排好课后延时服务,
成为社会各界
关注的焦点。另
一方面,
“双减”政策对学校课后服务的水平、课
堂教育教学的质量、教师参与课后服务等方面
的内容都提出了新要求 ,
但也意味着可能会给
教师带来更重的工作负担。例如,
广东省、浙江
省等地教育厅在推进课后服务水平的《实施意
见》中都提出 ,
在每周一到周五 ,
学校每天要至
少提供 2 个小时的课后服务 ,实行弹性离校制
度,
且对有特殊需要的学生提供延时托管服务。
这无形中延长了教师的工作时间 ,
增加了教师
的工作任务。于川在文章中提到 ,
“双减”政策
强化了学校的主阵地作用,
但也“容易引发教师
工作负担的连锁风险”,
且这种风险是“教育界
难以回避的现实问题”。\cite{ref3}

对于家庭而言,“双减”政策的目的并不仅仅是规范学科类
校外培训机构违规行为 ,
更本质的目的是针对
课外机构作为“影子教育”,
背后衍生出的教育
焦虑、教育资源不均衡以及学生、家庭负担过重
等深层次问题进行综合治理。但是 ,
“双减”政
策发布后,
家长态度分化。一方面,
一些家长对
“双减”政策非常看好且相当支持 ,
认为该项政
策的出台能够对校外培训机构乱象进行一系列
整顿 ,
规范市场秩序 ,
减轻家庭经济负担 ,
一定
程度上能够缓解家长的教育焦虑。另一方面 ,
一些家长对于“双减”政策是否能够真正平衡优
质教育资源、实现教育资源均衡配置持怀疑态
度。部分家长对于学校教育质量和教育内容存
在不信任现象 ,
担心学校教育难以使得学生达
到升学的要求 ;
也有家长担心取消学科校外培
训后,
会导致“一对一私教”兴起,
中下收入家庭更难以负担起昂贵的“一对一私教”费用 ,
子女
会更加落后于家庭条件好的孩子,
“不能输在起
跑线上”的传统教育思想引发新一轮的焦虑心
态。例如,
学者周序指出,
“双减”政策可能会引
发家长的群体性焦虑\cite{ref4} ,发展为一个严峻的社
会问题。家长因社会阶层纵向流动的诉求及资
本和培训结构的渲染 ,
对“减负”是否能够合理
配置教育资源、保证良好的教育质量始终持怀
疑态度,
对“双减”政策存在抵触心理。

由于对
教育的供给端教育机构和需求端家庭都造成
了巨大冲击,因此“双减“
政策落地的过程中,出现了多重现实问题是难以避免的。
随着各地陆续推进“双减”政策的落实工
作,
政府、教培行业、学校等政策利益相关者面
临的困境也逐步暴露出来。

“双减”政策虽然会抑制供给端培训机构的
数量和规模,
但从需求端来讲,
短期内家长对于
学科培训的需求很难明显被抑制。韩国曾经也
因教育内卷、家庭经济压力大、教育机会不公平
等原因取缔过校外学科培训机构 ,
但由此引发
的高收入家庭聘请原培训机构的老师当家教形
成的“别墅补习”现象,
使得该政策最终失败,
韩
国政府最终又恢复开放校外培训 ,
培训机构再
次实现空前的繁荣。当前有舆论观点认为,
“双
减”政策后 ,或许校外学科培训会复制韩国现
象,
以“个人工作室”
“一对一家教”
“游击培训”
等模式卷土重来,
这将为监管带来不小的挑战。
有些条件较优越的家庭 ,
在“双减”政策发布后
已经开始谋划搞“私人订制”
“住家教师”
“家庭
私塾”,
市场上“家教 O2O”模式也似乎有火热的
苗头。因此 ,
如何避免校外学科培训走上被取
缔后又恢复实现再次繁荣”的道路,
对或许会巧
立名目、走擦边球的校外学科培训新方式进行
有效监管 ,
成为未来政府部门需要关注的重点
问题之一。有学者也提出,
在政策执行实践中,
需要探索科学鉴定学科和非学科类别的校外培
训机构的标准 ,
防止机构打擦边球将学科类培
训包装为非学科类培训等违规经营行为 ,
同时
还要关注目前教育行政执法部门执法力量不
足、
监管能力有限等问题。\cite{ref5}

从学校角度看 ,学校要在校内实现教育教
学质量和服务水平进一步提升、作业布置更加
科学合理、学校课后延时服务基本满足学生需
要 、学 生 学 习 更 好 回 归 校 园 且 学 足 学 好 等 目
标,
需要更多的优秀教师、更丰富的教学资源 ,
面临巨大的教育服务供给压力。从教师角度
看 ,高质量的教学对教师自身素质要求较高 ,
但是教师教学任务繁重 ,
平衡工作和生活之余
还要进一步提升个人能力 ,
教师负担加剧。同
时 ,提供课后延时服务 ,也存在占用教师额外
时间的可能性 ,教师可支配时间减少 ,工作负
担更重。

“双减”政策的落实,
涉及多方利益主体,
面
临复杂的利益关系网络与利益群体博弈。在
“双减”政策中,
“各行动主体要改变既有的政府
‘包打天下’和市场万能的治理思路 ,
树立网络
化治理思维” \cite{ref6}。要想推进“双减”政策落地见
效,
打破原有的利益网络和观念藩篱,
形成健康
的教育生态。

一方面
需要政府发挥主导作用,
组织多元
主体力量,
合力推进政策有效落实。
需要地方政府和教育行政管理部门开展
多方位的努力:
第一 ,加强高位统筹。校外培训机构治理
涉及到十几个政府部门,
是一项综合治理工作,
必须从顶层设计入手,
加强高位统筹协调,
才能
顺利推进政策落地见效。
第二 ,严格监督制度。2021 年 11 月 3 日 ,
国家市场监管总局等 8 部门联合发布了《关于
做好校外培训广告管控的通知》,
要求严格清查
线上线下空间校外培训广告 ,
综合运用多种手
段确保“双减”政策要求落实到位。
第三,
开展科学监管。在大数据时代,
靠传
统的人工监管方式很难达到良好的监管效果 ,
需要依靠现代化信息技术赋能建立科学监管体
系。
第四,
推进长效治理。开展部门联合抽查、
处置投诉举报、
“24 小时在线监测”等日常执法
检查行动;
对于执法中发现的虚假宣传、违规收
费等乱象给予顶格罚款,
形成震慑效应。

另一方面要助推学校提升教学质量和服务内容,
消除家长教育焦虑。单方面的抑制供给并不能从根本上消除家
长的教育焦虑,
只有提升学校教育质量、让学生
在校内学足学好 ,
才能从根本上消解群体性教
育焦虑,
治理校外培训热。
第一 ,通过课堂增效促进学生“减负”。好
的教师、好的课程,
是课堂增效的关键所在。
第二,
通过课后服务提供差异化学习指导。
家长的焦虑在于对优质教育资源获得的不确定
性\cite{ref7} ,担心校内教育难以满足学生的差异化学
习能力和学习成效。
第三 ,优化课程设置发展学生综合素养。
在过去的教育中,
学生被“勤奋文化”所束缚,
闲
暇时间被剥夺。
\let\cleardoublepage\clearpage
\begin{thebibliography}{99}
    \bibitem{ref1} 余晖 .“双减”时代基础教育的公共性回归与公平性隐忧[J]. 南京社会科学,
    2021,
    (12).
    \bibitem{ref2} 朱益明 .“双减”:认知更新、制度创新与改革
    行动[J]. 南京社会科学,2021,
    (11).
    \bibitem{ref3} 于川,杨丽乐 .“双减”政策背景下教师工作
    负担的风险分析及其化解[EB/OL]. 当代教育论坛.

    \bibitem{ref4} 周序 . 家庭资本与学业焦虑——试论“双减”
    政策引发的家长焦虑问题[J]. 广西师范大学学报(哲
    学社会科学版),
    2021,(6).

    \bibitem{ref5} 马开剑,王光明,方芳,张冉,艾巧珍,李廷洲 .
    “双减”政策下的教育理念与教育生态变革(笔谈)[J].
    天津师范大学学报(社会科学版),2021,
    (6).

    \bibitem{ref6} 陈庆礼 .“双减”政策网络结构、利益博弈与
    治理路径[J]. 教师教育论坛,2021,(10).

    \bibitem{ref7} 刘复兴,董昕怡 . 实施“双减”政策的关键问
    题与需要处理好的矛盾关系[J]. 新疆师范大学学报
    (哲学社会科学版),
    2022,(1).
\end{thebibliography}

\end{document}
