%%%%% --------------------------------------------------------------------------------
%%
%%%%***************************** Literaure Review ***********************************
%%
%%% ++++++++++++++++++++++++++++++++++++++++++++++++++++++++++++++++++++++++++++++++++
%\part{\FakeBold{边界层理论文献综述}} % (fold)
%\label{prt:文献综述_}
\begin{abstract}
  \noindent \textbf{摘要:}  本文主要记录了边界层理论的起源时代背景和之后在渐进匹配中的系统发展。20世纪初Ludwig Prandtl提出边界层理论将流体力学分裂的两部分联合到一起,之后P.A.Lagerstrom和他的同事在边界层理论基础上建立了渐进匹配法,虽然渐进匹配法在粘性流体运动方程的进一步精确逼近上没有显著的实际效果。但是它的发展跳出了粘性流体方程的框架,在奇异摄动问题中发挥着重要的作用。\par\noindent \textbf{关键词:}  边界层理论;粘性流体力学方程;渐进匹配法;
\end{abstract}
\setcounter{section}{0}
\section{边界层理论的提出背景} % (fold)
\label{sec:背景介绍}
    流体力学中至关重要的粘性流体的运动方程Navier-Stokes方程组是在19世纪上半叶由Navier (1823),Poisson (1831),Saint-Venant (1843) 和 Stokes (1845)合作建立。但是由于粘性的存在和粘性本身随着相邻物体的物理性质会有较大区别的性质,在Navier-Stokes方程中产生了一些非线性项。如此一来因为在数学上非线性项导致积分困难使得人们不得不忽略非线性项,从而给出方程解只有在特殊情况(忽略流体粘性,即水和空气粘度很小,并因此产生的粘性力和剩余力很小)才成立。这在物理世界中是不符合普遍情况的,是脱离实际并且与物理实验结相违背的\cite{Tani1977}。\par
    
    如上矛盾造成流体力学发展到19世纪末已经分裂成两个几乎不相关的学科。一方面是忽略粘性的理想流体运动科学,发展地非常成熟,但是由于理想无粘性流体与日常经验的矛盾,使得其没有什么实际意义。另一方面是工程师们依赖于大量实验数据形成了一个经验学科——水力学。这两个学科的基础和方法都大相径庭。

	\section{边界层理论的提出} % (fold)
	\label{sec:国内外研究现状}

    上面所说的困境最终由Ludwig Prandtl打破,他做出的伟大成就是在这两个几乎无关的学科之间建立了联系。提出两个发散的方向的统一,因此同时在流体力学的理论研究和实验数据中都有非常大的相关同步。1904年的Heidelberg mathematical congress中,他的“几乎没有摩擦力的流体运动”的简短演讲中如此说道:“尽管对于理论流体运动而言,流体内部的摩擦力起着比较小的作用,但是无粘流体的理论和经验并不一致。如果假设摩擦常数非常小,但不是零,则当速度出现较大差异(例如,在固体壁上)时,其影响非常明显。当假设沿壁摩擦效应是一阶的时候,我们可以得到接近真实的结果,并且在自由流体中,我们仍假设是无粘流动\cite{Eckert2019}。”。

    这段话形象地展示了,流体可以分为两个区域: 一个很薄的层装区域(边界层(Boundary-Layer)),其中粘性很重要,另一个在该层之外的区域,粘性可以忽略不计。借助这一概念。不仅从物理上令人信服地解释了粘性在阻力问题中的重要性,而且通过极大地降低数学难度,为粘性流体的理论处理开辟了一条道路。Prandtl 边界层理论自20世纪初以来,已被怎么证明是非常有用的,并为流体力学的研究提供了相当大的激励。在蓬勃发展的飞行技术的影响下,新理论迅速发展,并很快成为现代流体力学的一个重点。

    虽然几乎所有流体力学书籍都将粘性边界层的概念归功于Prandtl,但是他本人十分谦逊,并归功于物理学家L.Lorenz的早期工作\cite{Lorenz1881},他写道:“这篇论文同时是关于自由热传导的第一篇论文,也是关于边界层的第一篇论文\cite{Prandtl1953}。”


    \section{边界层理论的发展}\label{sec:发展}
    之前提到Prandtl边界层理论给出了固体壁附近Navier-Stokes方程解的一阶近似。近似改变了方程的类型并降低了它的阶数,但是当试图改进它的时候出现了困难。Prandtl本人于1935年提出了通过修正位移厚度的影响来改进半无限平面板上的流动方程解的可能性\cite{Prandtl1935},还有其他作者考虑了壁面曲率、外部涡度、下游扰动等的影响。产生了一些特殊情况的有争议的结果。直到20世纪50年代Lagerstrom和他的同时才开始系统的研究。建立以Prandtl对Navier-Stokes方程组的近似为基础的Navier-Stokes的渐进解,从而得到现在所称的渐进匹配展开法。代表成果有Kaplun\cite{Lagerstrom1967}、Lagerstrom\cite{Lagerstrom1989}。

		\subsection{渐进匹配展开法}\label{sub:渐进匹配法}
 
        渐进匹配思想可以如下描述。以上Prandtl边界层理论将流体分为两个区域:有粘流体区域和无粘流体区域,然后在两个区域分别构造渐进展开式。求出的两个渐进展开式可以分别被称为内部近似展开式和外部近似展开式或者局部和全局近似展开式(这在边界层恰好包含全局区域时——正如处理的大部分边界层问题那样——是非常形象并且合适的术语),但是现在两个近似之间还没有建立合适的联系。在两个展开式的有效重叠区进行匹配即建立联系的过程又被称为渐进匹配(Kaplun在有效重叠区域的性质上做出了很多贡献)。

        对于Prandtl第一提到边界层是说道的一阶粘性边界层中进行匹配,可以应用术语“极限匹配原理”:全局近似的局部极限必须等于局部近似的全局极限。但在更微妙的情况下——例如,处理其他变量或更高节——可能需要使用中间变量或所谓的渐进匹配原理进行匹配。任何匹配方案的基本思想都是由近似侏儒和远视的塔克巨人形象地表达出来的\cite{Tuck1971}。

        \subsection{匹配原理}\label{sec:理论基础}
        匹配原理是匹配过程除了找到有效重叠区以外的另一个的核心,它在两个展开式的有效重叠区内要求两个展开式满足一些等式。而且有着不止一种匹配原理可供于选择使用,仅取决于研究者的需求和内外近似展开式的形式。如上所说最原始的渐进匹配原理即是Prandtl默认应用的极限匹配原理
        \begin{align*}
          \textmd{ 外部表达式的内部极限 } = \textmd{ 内部表达式的外部极限}
        \end{align*}
        这个基本规则是必要的或充分的,不仅取决于问题,还取决于所匹配的自变量选择。因此进一步我们可以通过更精确地描述被匹配数量的限制行为来改进这个原始的规则。我们用渐进代替单纯的极限。这就是匹配原理
        \begin{align*}
          \textmd{ 外部表达式在有效重叠区域的值 } 
        = \textmd{ 内部表达式在有效重叠区域的值 } 
        \end{align*}
        这里的内(外)表达式指的是内(外)变量的渐进展开式中第一个非零项。如果在渐进展开式中进一步保留项来将原理推广到更高的近似。我们得到渐进匹配原理
        \begin{align*}
          \textmd{ 第m项外部表达式在有效重叠区域的值 } = \textmd{ 第n项内部表达式在有效重叠区域的值 }
        \end{align*}
        这里m,n可以说任意两个整数;根据经验m的值经常是n和n+1。
        
        \subsection{渐进匹配展开的意义}\label{sec:匹配展开意义}
        然而,在给定的边界条件下,粘性解并不是唯一的,通常很难选择相应的解,即Navier-Stokes方程组解的极限。并且Murray,J.D\cite{Murray1965}对在半无限平面版上的流动进行了更高近似的计算,有趣的是结果与Prandtl\cite{Prandtl1935}的预期相反,二阶位移的影响消失了,不确定常数依赖于前缘附近的流动细节,在哪里边界层近似结果是无效的。不过Prandtl的预期在有限平板上是成立的,Kuo在有限平板上得到了非零的二阶位移修正项\cite{Kuo1953}。即使对于一般高雷诺数,边界层的二阶修正也很小,因此其计算主要是理论上的兴趣。然而值得注意的是,发现边界层理论的概念被拓展并推广到渐进匹配展开法,为处理微分方程的奇异摄动问题开辟了道路,相应成果见\cite{doi:10.1137/1036098}。因此,边界层理论的基本思想已被应用于流体力学以外的科学,并在流体力学中被应用与小粘度问题相关之外的其他问题。



        

        \newpage
        %\section*{参考文献} % (fold)
        %\label{sec:参考文献1}
		%\addcontentsline{toc}{section}{\protect\numberline{}参考文献}%
		%\begingroup
		    %\setlength{\bibsep}{0pt}
		    %\setstretch{1}
    \renewcommand\refname{参考文献}
		    \bibliographystyle{siam}
		    \bibliography{bib/lib}%
		%\endgroup
	% section* 参考文献 (end)
% part 文献综述_ (end)

%%% ++++++++++++++++++++++++++++++++++++++++++++++++++++++++++++++++++++++++++++++++++

%%% Local Variables:
%%% mode: latex
%%% TeX-master: "../main"
%%% End:
