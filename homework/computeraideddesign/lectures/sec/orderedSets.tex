\begin{defn}
  The \emph{Cartesian product} ${\cal X}\times {\cal Y}$
   between two sets ${\cal X}$ and ${\cal Y}$
   is the set of all possible ordered pairs with first element
   from ${\cal X}$ and second element from ${\cal Y}$:
   \begin{equation}
     {\cal X}\times {\cal Y} = \{(x,y)\ |\ x\in {\cal X},\ y\in {\cal Y}\}.
   \end{equation}
\end{defn}

\begin{axm}[Fundamental principle of counting]
  \label{axm:multiplicationPrinciple}
  A task consists of a sequence of $k$ independent steps.
  Let $n_i$ denote the number of different choices for the $i$-th step,
   the total number of distinct ways to complete the task
   is then
   \begin{equation}
     \prod_{i=1}^{k} n_i = n_1n_2\cdots n_k.
   \end{equation}
 \end{axm}

 \begin{exm}
   \label{exm:dinnerCombo}
   Let $A, E, D$ be the set of appetizers,
    main entrees, desserts in a restaurant.
   $A\times E\times D$
    is the set of possible dinner combos.
   If $\#A=10$, $\#E=5$, $\#D=6$,
    $\#(A\times E\times D)=300$.
 \end{exm}

 \begin{rem}
   The menu of a restaurant in Example \ref{exm:dinnerCombo}
    seldom contains all the different combinations
    explicitly spelled out.
   You simply pick one entry from each category
    and call the three choices your own dinner.
   In math learning,
    you follow a similar pattern.
   You learn algebra and topology,
    then you combine the two to learn algebraic topology.
   The difference is that,
    the process of coupling the two is much more difficult
    and much more powerful.
 \end{rem}

% \begin{defn}[Maximum and minimum]
%   Consider ${\cal S}\subseteq \mathbb{R}$,
%    ${\cal S}\ne \emptyset$.
%   If $\exists s_m\in{\cal S}$
%    s.t. $\forall x\in {\cal S}$, $x\le s_m$,
%    then $s_m$ is the \emph{maximum} of ${\cal S}$
%    and denoted by $\max {\cal S}$.
%   If $\exists s_m\in{\cal S}$
%    s.t. $\forall x\in {\cal S}$, $x\ge s_m$,
%    then $s_m$ is the \emph{minimum} of ${\cal S}$
%    and denoted by $\min {\cal S}$.
% \end{defn}

% \begin{defn}[Upper and lower bounds]
%   Consider ${\cal S}\subseteq \mathbb{R}$,
%    ${\cal S}\ne \emptyset$.
%   $a$ is an \emph{upper bound} of ${\cal S}\subseteq \mathbb{R}$
%   if $\forall x\in {\cal S}$, $x\le a$;
%    then the set ${\cal S}$ is said to be \emph{bounded above}.
%   $a$ is a \emph{lower bound} of ${\cal S}$
%    if $\forall x\in {\cal S}$, $x\ge a$;
%    then the set ${\cal S}$ is said to be \emph{bounded below}.
%   ${\cal S}$ is \emph{bounded}
%    if it is bounded above and bounded below.
% \end{defn}

% One difference between a maximum and an upper bound
%  is that the former belongs to the set
%  while the latter might not.
% Another difference is that,
%  for a bounded interval,
%  the upper bound always exists
%  while the maximum might not exist.

% \begin{defn}[Supremum and infimum]
%   Consider ${\cal S}\subseteq \mathbb{R}$,
%    ${\cal S}\ne \emptyset$.
%   If ${\cal S}$ is bounded above and ${\cal S}$
%    has a least upper bound 
%    then we call it the \emph{supremum}
%    of ${\cal S}$
%    and denote it by $\sup {\cal S}$.
%   If ${\cal S}$ is bounded below and ${\cal S}$
%    has a greatest lower bound,
%    then we call it the \emph{infimum}
%    of ${\cal S}$
%    and denote it by $\inf {\cal S}$.
% \end{defn}

% \begin{exm}
%   If ${\cal S}$ has a maximum, then $\max{\cal S}=\sup {\cal S}$.
% \end{exm}

% \begin{exm}
%   $\sup[a,b]=\sup[a,b)=\sup(a,b]=\sup(a,b)$.
% \end{exm}

% \begin{axm}[Completeness of $\mathbb{R}$]
%   Every nonempty subset of $\mathbb{R}$ %${\cal S}\subseteq \mathbb{R}$
%    that is bounded above has a least upper bound.
% \end{axm}

% In other words, 
%  for any nonempty ${\cal S}\subseteq \mathbb{R}$ bounded above,
%  $\sup {\cal S}$ exists and is a real number.

%  \begin{coro}
%   Every nonempty subset of $\mathbb{R}$ %${\cal S}\subseteq \mathbb{R}$
%    that is bounded below has a greatest lower bound.
%  \end{coro}

 \begin{defn}
   A \emph{binary relation between two sets} ${\cal X}$ and ${\cal Y}$
    is an ordered triple
    (${\cal X}, {\cal Y}, {\cal G}$)
    where ${\cal G}\subseteq{\cal X}\times{\cal Y}$.\\
    % or equivalently, a map
    % $R: {\cal X}\times{\cal Y}\rightarrow {\cal G}$.\\
  A \emph{binary relation on} ${\cal X}$
    is the relation between ${\cal X}$ and ${\cal X}$.\\
  The statement $(x,y)\in R$ is read
   ``$x$ is $R$-related to $y$,'' and
   denoted by $xRy$ or $R(x,y)$.
 \end{defn}

  \begin{defn}
   \label{defn:equivalenceRelation}
    An \emph{equivalence relation} ``$\sim$'' on ${\cal A}$ is 
    a binary relation on ${\cal A}$ 
    that satisfies
    $\forall a,b,c\in{\cal A}$,
    \begin{itemize}
      \itemsep0em
    \item $a\sim a$ (reflexivity);
    \item $a\sim b$ implies $b\sim a$ (symmetry);
    \item $a\sim b$ and $b\sim c$ imply $a\sim c$ (transitivity).
    \end{itemize}
  \end{defn}

 \begin{defn}
   \label{defn:totalOrder}
   A binary relation ``$\le$'' on some set ${\cal S}$
    is a \emph{total order} or \emph{linear order} on ${\cal S}$
    iff,
    $\forall a,b,c\in{\cal S}$,
    \begin{itemize}
    \item $a\le b$ and $b\le a$ imply $a=b$ (antisymmetry);
    \item $a\le b$ and $b\le c$ imply $a\le c$ (transitivity);
    \item $a\le b$ or $b\le a$ (totality).
    \end{itemize}
   A set equipped with a total order
    is a \emph{chain} or \emph{totally ordered set}.
 \end{defn}

 \begin{exm}
   The real numbers with less or equal.
 \end{exm}

 \begin{exm}
   The English letters of the alphabet with dictionary order.
 \end{exm}

 \begin{exm}
   The Cartesian product of a set of totally ordered sets
   with the \emph{lexicographical order}.
 \end{exm}

 \begin{exm}
   Sort your book in lexicographical order
    and save a lot of time.
   $\log_{26}N \ll N$!
 \end{exm}

 \begin{defn}
   \label{def:partialOrder}
   A binary relation ``$\le$'' on some set ${\cal S}$
    is a \emph{partial order} on ${\cal S}$
    iff, $\forall a,b,c\in{\cal S}$,
    antisymmetry, transitivity, and reflexivity ($a\le a$)
    hold.\\
   A set equipped with a partial order
    is a called a \emph{poset}.
 \end{defn}

 \begin{rem}
   Change the symmetry condition 
   in Definition \ref{defn:equivalenceRelation}
   to antisymmetry condition
   in Definition \ref{def:partialOrder}
   and you navigate from the equivalence relation
   to the partial order.
 \end{rem}

 \begin{exm}
   The set of subsets of a set ${\cal S}$
    ordered by inclusion ``$\subseteq$.''
   In this class we will not distinguish
    between ``$\subseteq$'' and ``$\subset$.''
 \end{exm}

 \begin{exm}
   The natural numbers equipped with the relation of divisibility.
 \end{exm}

 \begin{exm}
   The set of stuff you will put on your body every morning
    with the time ordered:
    undershorts, pants, belt, shirt, tie, jacket,
    socks, shoes, watch.
 \end{exm}

 \begin{exm}
   \label{exm:inheritance}
   Inheritance (``is-a'' relation) is a partial order.
   $A \rightarrow B$ reads ``$B$ is a special type of $A$''.
 \end{exm}

 \begin{exm}
   \label{exm:composition}
   Composition (``has-a'' relation) is also a partial order.
   $A \leadsto B$ reads
     ``B \emph{has an} instance/object of A.''
 \end{exm}

 \begin{exm}
   \label{exm:implication}
   Implication ``$\Rightarrow$''
    is a partial order on the set of logical statements.
 \end{exm}

 \begin{exm}
   The set of definitions, axioms,
    propositions, theorems, lemmas, etc., 
    is a poset with inheritance, composition, and implication.
   It is helpful to relate them with these partial orderings.\\
   ``If syntax sugar does not count, there is nothing left.''
 \end{exm}


%%% Local Variables:
%%% mode: latex
%%% TeX-master: "../notesAlgebraicTopology"
%%% End:
