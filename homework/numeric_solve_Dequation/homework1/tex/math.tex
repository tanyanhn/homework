\documentclass[a4paper]{book}

\usepackage{geometry}
% make full use of A4 papers
\geometry{margin=1.5cm, vmargin={0pt,1cm}}
\setlength{\topmargin}{-1cm}
\setlength{\paperheight}{29.7cm}
\setlength{\textheight}{25.1cm}

% auto adjust the marginals
\usepackage{marginfix}

\usepackage{amsfonts}
\usepackage{amsmath}
\usepackage{amssymb}
\usepackage{amsthm}
%\usepackage{CJKutf8}   % for Chinese characters
\usepackage{ctex}
\usepackage{enumerate}
\usepackage{graphicx}  % for figures
\usepackage{layout}
\usepackage{multicol}  % multiple columns to reduce number of pages
\usepackage{mathrsfs}  
\usepackage{fancyhdr}
\usepackage{subfigure}
\usepackage{tcolorbox}
\usepackage{tikz-cd}
\usepackage{listings}
\usepackage{xcolor} %代码高亮
\usepackage{braket}
%------------------
% common commands %
%------------------
% differentiation
\newcommand{\gen}[1]{\left\langle #1 \right\rangle}
\newcommand{\dif}{\mathrm{d}}
\newcommand{\difPx}[1]{\frac{\partial #1}{\partial x}}
\newcommand{\difPy}[1]{\frac{\partial #1}{\partial y}}
\newcommand{\Dim}{\mathrm{D}}
\newcommand{\avg}[1]{\left\langle #1 \right\rangle}
\newcommand{\sgn}{\mathrm{sgn}}
\newcommand{\Span}{\mathrm{span}}
\newcommand{\dom}{\mathrm{dom}}
\newcommand{\Arity}{\mathrm{arity}}
\newcommand{\Int}{\mathrm{Int}}
\newcommand{\Ext}{\mathrm{Ext}}
\newcommand{\Cl}{\mathrm{Cl}}
\newcommand{\Fr}{\mathrm{Fr}}
% group is generated by
\newcommand{\grb}[1]{\left\langle #1 \right\rangle}
% rank
\newcommand{\rank}{\mathrm{rank}}
\newcommand{\Iden}{\mathrm{Id}}

% this environment is for solutions of examples and exercises
\newenvironment{solution}%
{\noindent\textbf{Solution.}}%
{\qedhere}
% the following command is for disabling environments
%  so that their contents do not show up in the pdf.
\makeatletter
\newcommand{\voidenvironment}[1]{%
  \expandafter\providecommand\csname env@#1@save@env\endcsname{}%
  \expandafter\providecommand\csname env@#1@process\endcsname{}%
  \@ifundefined{#1}{}{\RenewEnviron{#1}{}}%
}
\makeatother

%---------------------------------------------
% commands specifically for complex analysis %
%---------------------------------------------
% complex conjugate
\newcommand{\ccg}[1]{\overline{#1}}
% the imaginary unit
\newcommand{\ii}{\mathbf{i}}
%\newcommand{\ii}{\boldsymbol{i}}
% the real part
\newcommand{\Rez}{\mathrm{Re}\,}
% the imaginary part
\newcommand{\Imz}{\mathrm{Im}\,}
% punctured complex plane
\newcommand{\pcp}{\mathbb{C}^{\bullet}}
% the principle branch of the logarithm
\newcommand{\Log}{\mathrm{Log}}
% the principle value of a nonzero complex number
\newcommand{\Arg}{\mathrm{Arg}}
\newcommand{\Null}{\mathrm{null}}
\newcommand{\Range}{\mathrm{range}}
\newcommand{\Ker}{\mathrm{ker}}
\newcommand{\Iso}{\mathrm{Iso}}
\newcommand{\Aut}{\mathrm{Aut}}
\newcommand{\ord}{\mathrm{ord}}
\newcommand{\Res}{\mathrm{Res}}
%\newcommand{\GL2R}{\mathrm{GL}(2,\mathbb{R})}
\newcommand{\GL}{\mathrm{GL}}
\newcommand{\SL}{\mathrm{SL}}
\newcommand{\Dist}[2]{\left|{#1}-{#2}\right|}

\newcommand\tbbint{{-\mkern -16mu\int}}
\newcommand\tbint{{\mathchar '26\mkern -14mu\int}}
\newcommand\dbbint{{-\mkern -19mu\int}}
\newcommand\dbint{{\mathchar '26\mkern -18mu\int}}
\newcommand\bint{
	{\mathchoice{\dbint}{\tbint}{\tbint}{\tbint}}
}
\newcommand\bbint{
	{\mathchoice{\dbbint}{\tbbint}{\tbbint}{\tbbint}}
}





%----------------------------------------
% theorem and theorem-like environments %
%----------------------------------------
\numberwithin{equation}{chapter}
\theoremstyle{definition}

\newtheorem{thm}{Theorem}[chapter]
\newtheorem{axm}[thm]{Axiom}
\newtheorem{alg}[thm]{Algorithm}
\newtheorem{asm}[thm]{Assumption}
\newtheorem{defn}[thm]{Definition}
\newtheorem{prop}[thm]{Proposition}
\newtheorem{rul}[thm]{Rule}
\newtheorem{coro}[thm]{Corollary}
\newtheorem{lem}[thm]{Lemma}
\newtheorem{exm}{Example}[chapter]
\newtheorem{rem}{Remark}[chapter]
\newtheorem{exc}[exm]{Exercise}
\newtheorem{frm}[thm]{Formula}
\newtheorem{ntn}{Notation}

% for complying with the convention in the textbook
\newtheorem{rmk}[thm]{Remark}


%\lstset{
%	backgroundcolor=\color{red!50!green!50!blue!50},%代码块背景色为浅灰色
%	rulesepcolor= \color{gray}, %代码块边框颜色
%	breaklines=true,  %代码过长则换行
%	numbers=left, %行号在左侧显示
%	numberstyle= \small,%行号字体
%	keywordstyle= \color{blue},%关键字颜色
%	commentstyle=\color{gray}, %注释颜色
%	frame=shadowbox%用方框框住代码块
%}
\lstset{
	columns=fixed,       
	numbers=left,                                        % 在左侧显示行号
	numberstyle=\tiny\color{gray},                       % 设定行号格式
	frame=none,                                          % 不显示背景边框
	backgroundcolor=\color[RGB]{245,245,244},            % 设定背景颜色
	keywordstyle=\color[RGB]{40,40,255},                 % 设定关键字颜色
	numberstyle=\footnotesize\color{darkgray},           
	commentstyle=\it\color[RGB]{0,96,96},                % 设置代码注释的格式
	stringstyle=\rmfamily\slshape\color[RGB]{128,0,0},   % 设置字符串格式
	showstringspaces=false,                              % 不显示字符串中的空格
	language=c++,                                        % 设置语言
}

%----------------------
% the end of preamble %
%----------------------

\begin{document}
\pagestyle{empty}
\pagenumbering{roman}

%\tableofcontents
%\clearpage

%\pagestyle{fancy}
%\fancyhead{}
%\lhead{Qinghai Zhang}
%\chead{Notes on Algebraic Topology}
%\rhead{Fall 2018}


\setcounter{chapter}{0}
\pagenumbering{arabic}
% \setcounter{page}{0}

% --------------------------------------------------------
% uncomment the following to remove these environments 
%  to generate handouts for students.
% --------------------------------------------------------
% \begingroup
% \voidenvironment{rem}%
% \voidenvironment{proof}%
% \voidenvironment{solution}%


% each chapter is factored into a separate file.

\chapter{数学文档}
%\begin{lstlisting}
%int main(){
%  double d;
%  int i;
%  return i;
%}
%\end{lstlisting}
% The main ingredients of snacks are sugar and fat;
%  the main ingredients of math are logic and set theory.

\section{需要解决的问题}
已有方程
\[- u^{''} = \exp(\sin(x))\sin(x) - \exp(\sin(x)) \cos^2(x) \qquad 0 < x < 1\]
\[u(0) = 1, u(1) = \exp(\sin(1))\]
希望求解$u(x) = \exp(\sin(x))$在$(0,1)$上的近似值达到一定的精确度.
\section{求解的方法}
\subsection{离散方程}
\begin{asm}
	$-v^{''}$ 可以离散为如下形式
	\[\frac{-v_{j-1} + 2 v_j + v_{j+1}}{h^2} \qquad 1 \leq j \leq n-1. \]
\end{asm}
因此可以将微分方程变为方程组
\[ \frac{1}{h^2} A \mathbf{v} = f(\mathbf{v})\]
其中$\mathbf{v}$为$u$在离散格点上的近似值.$A$为$(n-1)\times(n-1)$阶已知矩阵.

\subsection{方程组求解}
\subsubsection{迭代}
方程组求解有迭代法,
\begin{defn}
	Jacobi迭代
	\[v_j^* = \frac{1}{2}(v_{j-1}^{(0)} + v_{j+1}^{(0)} + h^2 f_j) \qquad 1 \leq j \leq n-1,\]
	
	加权Jacobi迭代
	\[v^{(1)} = (1 - w)v_j^{(0)} + w v_j^*  \qquad 1 \leq j \leq n-1 \].
	
	Gauss-Sediel迭代
	\[v_j = \frac{1}{2}(v_{j-1} + v_{j+1} + h^2 f_j) \qquad 1 \leq j \leq n-1\]
	$j$从$1 \rightarrow n-1$依次进行
	
	red-black Gauss-Sediel迭代
	
	首先计算
	\[v_{2j} = \frac{1}{2}(v_{2j-1} + v_{2j+1} + h^2 f_{2j}),\]
	然后
	\[v_{2j+1} = \frac{1}{2}(v_{2j} + v_{2j+2} + h^2 f_{2j+1}).\]
\end{defn}

\subsubsection{多重网格}
为了让值$\mathbf{v}$在不同网格上迭代松弛,需要定义从细网格到粗网格的限制算子$I_h^{2h}$,和粗网格到细网格的插值算子$I_{2h}^h$.
\begin{defn}
	插值算子有线性插值如下
	\[v_{2j}^h = v_j^{2h}, \]
	\[v_{2j+1}^h = \frac{1}{2}(v_j^{2h} + v_{j+1}^{2h}),\]
	\[0 \leq j \leq \frac{n}{2} - 1\]
\end{defn}
\begin{defn}
	限制算子有全权重限制如下
\[v_j^{2h} = \frac{1}{4}(v_{2j-1}^h + 2 v_{2j}^h + v_{2j+1}^h),\]
	\[0 \leq j \leq \frac{n}{2} - 1\]
	插入限制如下
	\[v_j^{2h} = v_{2h}^h.\]
\end{defn}

不同粗细网格之间的传递松弛方式产生不同的循环策略,这里采用V-Cycle.

\begin{defn}
	定义每个V-Cycle为函数形式,满足输入初始值$\mathbf{v}$和右端值$\mathbf{f}$.输出近似值$\mathbf{V}$
	\[\mathbf{V} \leftarrow V^h(\mathbf{V}^h, \mathbf{f}^h). \]
	
	1. 使用输入的$\mathbf{v}^h$在方程组$A^h \mathbf{u}^h = \mathbf{f}^h$迭代松弛$\nu_1$次.
	
	
	2. 如果当前网格是最粗网格, 跳到第4步. 否则运行
	\begin{align*}
	\mathbf{f}^{2h} &\leftarrow I_h^{2h}(\mathbf{f}^h - A^h \mathbf{v}^h),\\
	\mathbf{v}^{2h} &\leftarrow 0, \\
	\mathbf{v}^{2h} &\leftarrow V^{2h}(\mathbf{v}^{2h}, \mathbf{f}^2h).
	\end{align*}
	
	3. 使用第2步的输出修正初始值
	\[\mathbf{v}^h \leftarrow \mathbf{v}^h + I_{2h}^{h} \mathbf{v}^{2h}.\]
	
	4. 再将$\mathbf{v}^h$ 松弛迭代 $\nu_2$次后输出$\mathbf{v}^h$.
	
	
\end{defn}


类似有FMG算法

\begin{defn}
	\[\mathbf{v}^h \leftarrow FMG^h(\mathbf{f}^h)\].
	
	1. 如果已经到了最粗网格,令$\mathbf{v}^h \leftarrow 0$ 并且跳到第3步.或者运行
	\begin{align*}
	\mathbf{f}^{2h} &\leftarrow I_h^{2h} (\mathbf{f}^h),\\
	\mathbf{v}^{2h} &\leftarrow FMG^{2h}(\mathbf{f}^2h).
	\end{align*}
	
	2. 使用第1步的输出插值得到
	\[\mathbf{v}^h\leftarrow I_{2h}^h . \]
	
	3. 使用2输出的$\mathbf{v}^h$和输入的$\mathbf{f}^h$作为初始值调用$\nu_0$次V-Cycle.
\end{defn}


结合多重网格和松弛迭代算法得求解算法.


\section{误差分析}
不妨将所有误差假设为正弦函数形式在每个网格点上的大小,定义误差
\[w_{k,j}^h = \sin\left(\frac{jk \pi}{n}\right), \qquad 1 \leq k \leq n-1, \quad 0 \leq j \leq n.\]

考虑迭代和网格之间插值限制对在网格点上的误差的影响.

\begin{thm}
	全权重限制算子
	\[I_h^{2h} \mathbf{w}_k^h = \cos^2\left(\frac{k\pi}{2n}\right) \mathbf{w}_k^{2h}, \qquad 1 \leq k \leq \frac{n}{2}.\]
	\[I_h^{2h} \mathbf{w}_{k^{'}}^h = - \sin^2\left(\frac{k\pi}{2n}\right) \mathbf{w}_k^{2h}, \qquad 1 \leq k \leq \frac{n}{2}.\]
	\[k^{'} = n - k.\]
	
	线性插值算子
	\[I_{2h}^h \mathbf{w}_k^{2h} = c_k \mathbf{w}_k^h - s_k \mathbf{w}_{k^{'}}^h, 
	\qquad 1 \leq k \le \frac{n}{2}, \quad k^{'} = n - k.\]
	\[c_k = \cos^2\left(\frac{k\pi}{2n}\right), s_k = \sin^2\left(\frac{k\pi}{2n}\right)\]
	
\end{thm}

在二重网格上的松弛,结合松弛,插值算子和限制算子可能运算$TG$, $TG$作用在$\mathbf{w}$上有

\begin{thm}
	\[TG \begin{bmatrix}
	\mathbf{w}_k \\
	\mathbf{w}_{k^{'}}
	\end{bmatrix} = 
	\begin{bmatrix}
	\lambda^{\nu_1 + \nu_2}_k s_k & \lambda^{\nu_1}_k \lambda^{\nu_2}_{k^{'}} s_k \\
	\lambda^{\nu_1}_{k^{'}} \lambda^{\nu_2}_{k} c_k  & \lambda^{\nu_1 + \nu_2}_{k^{'}} c_k
	\end{bmatrix} 
	\begin{bmatrix}
	\mathbf{w}_k \\ 
	\mathbf{w}_{k^{'}}
	\end{bmatrix}
	=:
	\begin{bmatrix}
	c1 & c2 \\
	c3 & c4
	\end{bmatrix}
	\begin{bmatrix}
	\mathbf{w}_k \\ 
	\mathbf{w}_{k^{'}}
	\end{bmatrix}  \qquad  1 \leq k \leq n/2\]
	
	松弛方法决定了$\lambda_k^{'}$较小, $s_k = \sin^2\left(\frac{k\pi}{2n}\right)$当$k \leq n/2$ 较小.因此每次$TG$算子都是将误差项$\mathbf{w}$缩小.
\end{thm}


\end{document}


%%% Local Variables: 
%%% mode: latex
%%% TeX-master: t
%%% End: 

% LocalWords:  FPN underflows denormalized FPNs matlab eps IEEE iff
% LocalWords:  cardinality significand quadratically bijection unary
%  LocalWords:  contractive bijective postcondition invertible arity
%  LocalWords:  subspaces surjective injective monomials additivity
%  LocalWords:  nullary Abelian abelian finitary eigenvectors adjoint
%  LocalWords:  eigenvector nullspace Hermitian unitarily multiset
%  LocalWords:  nonsingular nonconstant homomorphism homomorphisms
%  LocalWords:  isomorphically indeterminates subfield isomorphism
%  LocalWords:  nondefective diagonalizable contrapositive cofactor
%  LocalWords:  submatrix nilpotent positivity orthonormal extremum
%  LocalWords:  Jacobian nonsquare semidefinite nonnegative RHS LLS
%  LocalWords:  roundoff closedness
