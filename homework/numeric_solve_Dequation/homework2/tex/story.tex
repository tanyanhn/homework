\documentclass[a4paper]{book}

\usepackage{geometry}
% make full use of A4 papers
\geometry{margin=1.5cm, vmargin={0pt,1cm}}
\setlength{\topmargin}{-1cm}
\setlength{\paperheight}{29.7cm}
\setlength{\textheight}{25.1cm}

% auto adjust the marginals
\usepackage{marginfix}

\usepackage{amsfonts}
\usepackage{amsmath}
\usepackage{amssymb}
\usepackage{amsthm}
%\usepackage{CJKutf8}   % for Chinese characters
\usepackage{ctex}
\usepackage{enumerate}
\usepackage{graphicx}  % for figures
\usepackage{layout}
\usepackage{multicol}  % multiple columns to reduce number of pages
\usepackage{mathrsfs}  
\usepackage{fancyhdr}
\usepackage{subfigure}
\usepackage{tcolorbox}
\usepackage{tikz-cd}
\usepackage{listings}
\usepackage{xcolor} %代码高亮
\usepackage{braket}
%------------------
% common commands %
%------------------
% differentiation
\newcommand{\gen}[1]{\left\langle #1 \right\rangle}
\newcommand{\dif}{\mathrm{d}}
\newcommand{\difPx}[1]{\frac{\partial #1}{\partial x}}
\newcommand{\difPy}[1]{\frac{\partial #1}{\partial y}}
\newcommand{\Dim}{\mathrm{D}}
\newcommand{\avg}[1]{\left\langle #1 \right\rangle}
\newcommand{\sgn}{\mathrm{sgn}}
\newcommand{\Span}{\mathrm{span}}
\newcommand{\dom}{\mathrm{dom}}
\newcommand{\Arity}{\mathrm{arity}}
\newcommand{\Int}{\mathrm{Int}}
\newcommand{\Ext}{\mathrm{Ext}}
\newcommand{\Cl}{\mathrm{Cl}}
\newcommand{\Fr}{\mathrm{Fr}}
% group is generated by
\newcommand{\grb}[1]{\left\langle #1 \right\rangle}
% rank
\newcommand{\rank}{\mathrm{rank}}
\newcommand{\Iden}{\mathrm{Id}}

% this environment is for solutions of examples and exercises
\newenvironment{solution}%
{\noindent\textbf{Solution.}}%
{\qedhere}
% the following command is for disabling environments
%  so that their contents do not show up in the pdf.
\makeatletter
\newcommand{\voidenvironment}[1]{%
  \expandafter\providecommand\csname env@#1@save@env\endcsname{}%
  \expandafter\providecommand\csname env@#1@process\endcsname{}%
  \@ifundefined{#1}{}{\RenewEnviron{#1}{}}%
}
\makeatother

%---------------------------------------------
% commands specifically for complex analysis %
%---------------------------------------------
% complex conjugate
\newcommand{\ccg}[1]{\overline{#1}}
% the imaginary unit
\newcommand{\ii}{\mathbf{i}}
%\newcommand{\ii}{\boldsymbol{i}}
% the real part
\newcommand{\Rez}{\mathrm{Re}\,}
% the imaginary part
\newcommand{\Imz}{\mathrm{Im}\,}
% punctured complex plane
\newcommand{\pcp}{\mathbb{C}^{\bullet}}
% the principle branch of the logarithm
\newcommand{\Log}{\mathrm{Log}}
% the principle value of a nonzero complex number
\newcommand{\Arg}{\mathrm{Arg}}
\newcommand{\Null}{\mathrm{null}}
\newcommand{\Range}{\mathrm{range}}
\newcommand{\Ker}{\mathrm{ker}}
\newcommand{\Iso}{\mathrm{Iso}}
\newcommand{\Aut}{\mathrm{Aut}}
\newcommand{\ord}{\mathrm{ord}}
\newcommand{\Res}{\mathrm{Res}}
%\newcommand{\GL2R}{\mathrm{GL}(2,\mathbb{R})}
\newcommand{\GL}{\mathrm{GL}}
\newcommand{\SL}{\mathrm{SL}}
\newcommand{\Dist}[2]{\left|{#1}-{#2}\right|}

\newcommand\tbbint{{-\mkern -16mu\int}}
\newcommand\tbint{{\mathchar '26\mkern -14mu\int}}
\newcommand\dbbint{{-\mkern -19mu\int}}
\newcommand\dbint{{\mathchar '26\mkern -18mu\int}}
\newcommand\bint{
	{\mathchoice{\dbint}{\tbint}{\tbint}{\tbint}}
}
\newcommand\bbint{
	{\mathchoice{\dbbint}{\tbbint}{\tbbint}{\tbbint}}
}





%----------------------------------------
% theorem and theorem-like environments %
%----------------------------------------
\numberwithin{equation}{chapter}
\theoremstyle{definition}

\newtheorem{thm}{Theorem}[chapter]
\newtheorem{axm}[thm]{Axiom}
\newtheorem{alg}[thm]{Algorithm}
\newtheorem{asm}[thm]{Assumption}
\newtheorem{defn}[thm]{Definition}
\newtheorem{prop}[thm]{Proposition}
\newtheorem{rul}[thm]{Rule}
\newtheorem{coro}[thm]{Corollary}
\newtheorem{lem}[thm]{Lemma}
\newtheorem{exm}{Example}[chapter]
\newtheorem{rem}{Remark}[chapter]
\newtheorem{exc}[exm]{Exercise}
\newtheorem{frm}[thm]{Formula}
\newtheorem{ntn}{Notation}

% for complying with the convention in the textbook
\newtheorem{rmk}[thm]{Remark}


%\lstset{
%	backgroundcolor=\color{red!50!green!50!blue!50},%代码块背景色为浅灰色
%	rulesepcolor= \color{gray}, %代码块边框颜色
%	breaklines=true,  %代码过长则换行
%	numbers=left, %行号在左侧显示
%	numberstyle= \small,%行号字体
%	keywordstyle= \color{blue},%关键字颜色
%	commentstyle=\color{gray}, %注释颜色
%	frame=shadowbox%用方框框住代码块
%}
\lstset{
	columns=fixed,       
	numbers=left,                                        % 在左侧显示行号
	numberstyle=\tiny\color{gray},                       % 设定行号格式
	frame=none,                                          % 不显示背景边框
	backgroundcolor=\color[RGB]{245,245,244},            % 设定背景颜色
	keywordstyle=\color[RGB]{40,40,255},                 % 设定关键字颜色
	numberstyle=\footnotesize\color{darkgray},           
	commentstyle=\it\color[RGB]{0,96,96},                % 设置代码注释的格式
	stringstyle=\rmfamily\slshape\color[RGB]{128,0,0},   % 设置字符串格式
	showstringspaces=false,                              % 不显示字符串中的空格
	language=c++,                                        % 设置语言
}

%----------------------
% the end of preamble %
%----------------------

\begin{document}
\pagestyle{empty}
\pagenumbering{roman}

%\tableofcontents
%\clearpage

%\pagestyle{fancy}
%\fancyhead{}
%\lhead{Qinghai Zhang}
%\chead{Notes on Algebraic Topology}
%\rhead{Fall 2018}


\setcounter{chapter}{0}
\pagenumbering{arabic}

\chapter{Time Integrator Story}

\section{原问题}
从三体问题方程组
\begin{align*}
	&u_1^{'} = u_4, \\
	&u_2^{'} = u_5, \\
	&u_3^{'} = u_6, \\
	&u_4^{'} = 2 * u_5 + u_1 - \frac{\mu (u_1 + \mu - 1)}{(u_2^2 + u_3^2 + (u_1 + \mu - 1)^2)^{3/2}} - 
	\frac{(1 - \mu)(u_1 + \mu)}{(u_2^2 + u_3^2 + (u_1 + \mu)^2)^{3/2}}, \\
	&u_5^{'} = -2 * u_4 +u_2 - \frac{\mu u_2}{(u_2^2 + u_3^2 + (u_1 + \mu - 1)^2)^{3/2}} - 
	\frac{(1 - \mu)u_2}{(u_2^2 + u_3^2 + (u_1 + \mu)^2)^{3/2}}, \\
	&u_6^{'} = - \frac{\mu u_3}{(u_2^2 + u_3^2 + (u_1 + \mu - 1)^2)^{3/2}} - 
	\frac{(1 - \mu)u_3}{(u_2^2 + u_3^2 + (u_1 + \mu)^2)^{3/2}}, \\
\end{align*}

不妨定义$\mathbf{f}$为右端项.给定两个初始值

\begin{align*}
	U1 := (u_1, u_2, u_3, u_4, u_5, u_6) &= (0.994, 0, 0, 0, −2.0015851063790825224, 0) \\
	U2 := (u_1, u_2, u_3, u_4, u_5, u_6) &= (0.87978, 0, 0, 0, −0.3797, 0)
\end{align*}

\section{理论支持}
如上即是一个(IVP)初值问题, 将方程组左端的$\mathbf{u}_t$,按$t$离散得

\begin{align*}
	\mathbf{u}(t_{n+1}) = \mathbf{u}(t_{n}) + k * \mathbf{f}
\end{align*}
$k$ 为时间步长.

\subsection{LMM方法}

拓展该离散方程,增加使用的已知的时间上$\mathbf{u}$的近似值.得(LMM)线性多步法.

\begin{defn}
	对于求解IVP问题的LMM方法即
	\[\sum_{j=0}^{s} \alpha_j \mathbf{U}^{n+j} = k \sum_{j=0}^{s} \beta_j \mathbf{f} (\mathbf{U}^{n+j}, t_{n+j}),\]
\end{defn}

对于LMM方法,有如下定理保证精度.

\begin{thm}
	LMM方法的单步误差服从
	\[\mathcal{L} \mathbf{u}(t_n) = C_0 \mathbf{u}(t_n) + C_1 k \mathbf{u}_t(t_n) +
	C_2 k ^2 \mathbf{u}_{tt}(t_n) + \dots ,\]
	这里
	\begin{align*}
		&C_0 = \sum_{j=0}^{s} \alpha_j \\
		&C_1 = \sum_{j=0}^{s} (j \alpha_j - \beta_j) \\
		&C_2 = \sum_{j=0}^{s} \left(\frac{1}{2} j^2 \alpha_j - j \beta_j\right) \\
		&\vdots \\
		&C_q = \sum_{j=0}^{s} \left(\frac{1}{q !} j^q \alpha_j - \frac{1}{(q-1) !} j^{q-1} \beta_j \right).
	\end{align*}
\end{thm}

采取不同的$\alpha \beta$的取法,并提高精度$p$,即可分别得到 Adams-Bashforth, Adams-Moulton, BDFs等特殊的LMM方法.

并且有定理协助保证LMM方法收敛性和,稳定性.

\begin{thm}
	一个LMM方法是0稳定的,当且仅当$\rho(x)$的所有根$z$满足$\left|z\right| \leq 1$,并且满足$\left|z\right| = 1$的根都不是重根.
	这里
	\[\rho(x) = \sum_{j=0}^{s} \alpha_j x^j. \]
\end{thm}

\begin{defn}
	一个LMM方法若阶$p \geq 1$是一致的.
\end{defn}

\begin{thm}
	一个LMM方法当且仅当一致并且0稳定的时是收敛的.
\end{thm}

并且误差满足
\begin{thm}
	对于一个IVP问题,右端项$\mathbf{f}$对$u,t$都满足$p$阶连续可导.那么对于一个收敛的$p$阶LMM方法和满足如下条件的初始条件
	\[\forall i = 0, 1, \ldots, s-1, \qquad \lVert \mathbf{U}^i - \mathbf{u}(t_i) \rVert = O(k^p),\]
	那么这个IVP的数值解满足当$k>0$充分小时
	\[\lVert \mathbf{U}^{t/k} - \mathbf{u}(t) \rVert = O(k^p)\qquad \forall t \in [0,T]. \]
\end{thm}

\subsection{Runge-Kutta 方法}

当不考虑使用时间较远的信息时,考虑在单步中多次计算提高精度,如此考虑阶产生了Runge-Kutta 方法.
\begin{defn}
	一个$s$步的显式Runge-Kutta方法是一种如下形式的单步法.
	\begin{equation}
		\left\{
		\begin{aligned}
		y_1 &= f(U^n, t_n), \\
		y_2 &= f(U^n + k a_{2,1} y_1, t_n + c_2 k), \\
		y_3 &= f(U^n + k(a_{3,1} y_1 + a_{3,2} y_2), t_n + c_3 k) ,\\
		& \ldots \\
		y_s &= f(U^n + k(a_{s,1}y_1 + a_{s,2} y_2 + \dots + a_{s,s-1} y_{s-1}), t_n + c_s k), \\
		U^{n+1} &= U^n + k(b_1 y_1 + b_2 y_2 + \ldots b_s y_s) =: 
		U^n + \Phi(U^n, t_n; k),
		\end{aligned}
		\right.
	\end{equation}
	这里$a_{i,j}, b_i,c_i$都是实数,并且 
	\[\forall i = 1, 2, \ldots,s, \qquad c_i = \sum_{j=0}^{i} a_{i,j}.\]
\end{defn}

当$s = 2$可以特化为Euler method, 当$s = 4$时特化为classical fourth-order Runge-Kutta method.即之后程序展示中使用的两种方法.

\begin{defn}
	modified Euler method 是一种有如下形式的单步法
	\begin{equation}
	\left\{
	\begin{aligned}
	y_1 &= f(U^n, t_n), \\
	y_2 &= f(U^n + \frac{k}{2} y_1, t_n + \frac{k}{2}), \\
	U^{n+1} &= U^n + k y_2 .
	\end{aligned}
	\right.
	\end{equation}
\end{defn}

]\begin{defn}
	Runge-Kutta method 是一种有如下形式的单步法
	\begin{equation}
	\left\{
	\begin{aligned}
	y_1 &= f(U^n, t_n), \\
	y_2 &= f(U^n + \frac{k}{2} y_1, t_n + \frac{k}{2}), \\
	y_3 &= f(U^n + \frac{k}{2} y_2, t_n + \frac{k}{2}), \\
	y_4 &= f(U^n + k y_3, t_n + k), \\
	U^{n+1} &= U^n + \frac{k}{6}(y_1 + 2 y_2 + 2 y_3 + y_4).
	\end{aligned}
	\right.
	\end{equation}
\end{defn}


与LMM方法类似,Runge-Kutta method 也有定理保持consistent, stable 和 convergence.

\begin{thm}
	一个Runge-Kutta method是consistent, 当且仅当若$(u,t)$在$f$的定义域中时满足
	\[\lim_{k\rightarrow 0} \Phi(u,t;k) = f(u,t)\]
\end{thm}

并且可以怎么Euler method的精确度为2阶,classical Runge-Kutta method 精确度为5阶..


\section{程序实现结果}

make run将会输出modified Euler method和classical Runge-Kutta method 在两个初值上不同步数计算的结果的误差和所用时间.可以看出 Runge-Kutta method 误差会小很多,并且计算所用时间只多一倍.

将它们的结果画出,如下图所示,其中红线是2400000步modifiedEuler法的计算结果作为精确解.黄线为计算的近似解其中Euler method为24000步,Runge-Kutta method 为 6000步.


\begin{figure}
	\centering
	\includegraphics[width=18cm,height=25cm]{../output/initial1.png}
	\caption{第一组初始值的数值结果}
\end{figure}


\begin{figure}
	\centering
	\includegraphics[width=18cm,height=25cm]{../output/initial2.png}
	\caption{第二组初始值的数值结果}
\end{figure}

\end{document}


%%% Local Variables: 
%%% mode: latex
%%% TeX-master: t
%%% End: 

% LocalWords:  FPN underflows denormalized FPNs matlab eps IEEE iff
% LocalWords:  cardinality significand quadratically bijection unary
%  LocalWords:  contractive bijective postcondition invertible arity
%  LocalWords:  subspaces surjective injective monomials additivity
%  LocalWords:  nullary Abelian abelian finitary eigenvectors adjoint
%  LocalWords:  eigenvector nullspace Hermitian unitarily multiset
%  LocalWords:  nonsingular nonconstant homomorphism homomorphisms
%  LocalWords:  isomorphically indeterminates subfield isomorphism
%  LocalWords:  nondefective diagonalizable contrapositive cofactor
%  LocalWords:  submatrix nilpotent positivity orthonormal extremum
%  LocalWords:  Jacobian nonsquare semidefinite nonnegative RHS LLS
%  LocalWords:  roundoff closedness
