\begin{defn}
  A \emph{set} ${\cal S}$
  is a collection of \emph{distinct} objects $x$'s,
   often denoted with the following notation
   \begin{equation}
     \label{eq:setNotation}
     {\cal S} = \{ x\ |\ \text{ the conditions that $x$ satisfies. } \}.
   \end{equation}
\end{defn}

\begin{ntn}
$\mathbb{R}, \mathbb{Z}, \mathbb{N}, \mathbb{Q}, \mathbb{C}$
 denote 
 the sets of real numbers, integers, natural numbers,
 rational numbers and complex numbers, respectively.
$\mathbb{R}^+, \mathbb{Z}^+, \mathbb{N}^+, \mathbb{Q}^+$
 the sets of positive such numbers.
\end{ntn}

 \begin{defn}
   ${\cal S}$ is a \emph{subset} of ${\cal U}$,
    written as ${\cal S}\subseteq {\cal U}$,
   if and only if (iff) $x\in {\cal S}$ $\Rightarrow$ $x\in {\cal U}$.
   ${\cal S}$ is a \emph{proper subset} of ${\cal U}$,
    written as ${\cal S}\subset {\cal U}$,
    if ${\cal S}\subseteq {\cal U}$
    and $\exists x\in {\cal U}$ s.t. $x\not\in{\cal S}$.
 \end{defn}

\begin{defn}[Statements of first-order logic]
\label{def:uni_exist}
A \emph{universal statement} is a logic statement 
 of the form
\begin{equation}
  \mathsf{U} = (\forall x\in {\cal S},\ \mathsf{A}(x) ).
\end{equation}
An \emph{existential statement} has the form
\begin{equation}
  \mathsf{E} = (\exists x\in {\cal S},\text{ s.t. } \mathsf{A}(x)),
\end{equation}
 where 
 $\forall$ (``for each'') and $\exists$ (``there exists'')
 are the \emph{quantifiers}, ${\cal S}$ is a set,
 ``s.t.'' means ``such that,''
 and $\mathsf{A}(x)$ is the \emph{formula}.\\
A statement of \emph{implication/conditional}
 has the form
 \begin{equation}
   \mathsf{A}\Rightarrow \mathsf{B}.
 \end{equation}
\end{defn}

 \begin{exm}
   Universal and existential statements:\\
   $\forall x\in[2,+\infty)$, $x>1$;\\
   $\forall x\in \mathbb{R}^+$, $x>1$;\\
   $\exists p,q\in \mathbb{Z}, \text{ s.t. } p/q = \sqrt{2}$;\\
   $\exists p,q\in \mathbb{Z}, \text{ s.t. } \sqrt{p} = \sqrt{q}+1$.
 \end{exm}

\begin{defn}
  \emph{Uniqueness quantification}
   or \emph{unique existential quantification},
   written $\exists!$ or $\exists_{=1}$, 
   indicates that exactly one object with a certain property exists.
\end{defn}

\begin{exc}
  Express the logic statement $\exists! x, \text{ s.t. } \mathsf{A}(x)$
   with $\exists$, $\forall$, and $\Leftrightarrow$.
\end{exc}
\begin{solution}
  $\exists x \text{ s.t. }\forall y, \mathsf{A}(y) \Leftrightarrow x=y.$
\end{solution}

 \begin{rem}
A logic statement is either true or false.
There is no such thing that
 a logic statement is sometimes true and sometimes false.
To prove a universal statement,
 conceptually we have to verify the statement
 for all elements in the set.
To deny a universal statement,
 we only need to show a counterexample.
To prove an existential statement,
 we only need to show an instance.
To deny an existential statement,
 conceptually we have to show that the statement holds
 for none of the elements.
 \end{rem}

 \begin{rem}
   In Definition \ref{def:uni_exist},
    the formula $\mathsf{A}(x)$ itself
    might also be a logic statement.
   Hence universal and existential statements
    might be nested.
   This observation leads to the next definition.
 \end{rem}

 \begin{defn}
   A \emph{universal-existential statement} is a logic statement 
   of the form
   \begin{equation}
     \mathsf{U}_E =
     (\forall x\in {\cal S},\ \exists y\in {\cal T}
     \text{ s.t. } \mathsf{A}(x,y)).
   \end{equation}
   An \emph{existential-universal statement} has the form
   \begin{equation}
     \mathsf{E}_U =
     (\exists y\in {\cal T},\text{ s.t. } \forall x\in {\cal S},\ 
     \mathsf{A}(x,y)).
   \end{equation}
 \end{defn}

 \begin{exm}
   True or false:\\
   $\forall x\in[2,+\infty)$, $\exists y\in \mathbb{Z}^+$ s.t. $x^y<10^5$;\\
   $\exists y\in \mathbb{R}$ s.t.
   $\forall x\in[2,+\infty)$, $x>y$;\\
   $\exists y\in \mathbb{R}$ s.t.
   $\forall x\in[2,+\infty)$, $x<y$.
 \end{exm}

\begin{exc}
  [Translating an English statement into a logic statement]
  Goldbach's conjecture states
   \emph{every even natural number greater than 2
     is the sum of two primes}.
 \end{exc}
\begin{solution}
  Let $\mathbb{P}\subset \mathbb{N}^+$
   denote the set of prime numbers.
  Then Goldbach's conjecture is
  \begin{displaymath}
  \forall a\in 2\mathbb{N}^++2,
   \exists p,q\in \mathbb{P} \text{ s.t. } a=p+q.\qedhere
  \end{displaymath}
\end{solution}

\begin{thm}
   The existential-universal statement
    implies the corresponding universal-existential statement,
    but not vice versa.
 \end{thm}

 \begin{exm}[Translating a logic statement to an English statement]
   Let ${\cal S}$ be the set of all human beings.\\
   $U_E=$($\forall p\in{\cal S}, \exists q\in{\cal S}$ s.t. $q$ is $p$'s mom.)
   \\
   $E_U$=( $\exists q\in{\cal S}$ s.t.
   $\forall p\in{\cal S}, $ $q$ is $p$'s mom.)\\
   $U_E$ is probably true, but $E_U$ is certainly false. \\
   If $E_U$ were true, then $U_E$ would be true. why?
 \end{exm}

\begin{axm}[First-order negation of logical statements]
  The negations of the statements in Definition \ref{def:uni_exist}
  are
  \begin{align}
  \neg \mathsf{U} &= (\exists x\in {\cal S},\text{ s.t. }
  \neg \mathsf{A}(x)).
  \\
  \neg \mathsf{E} &= (\forall x\in {\cal S},\ 
  \neg \mathsf{A}(x)).
  \end{align}
\end{axm}

\begin{rul}
  The negation of a more complicated logic statement
   abides by the following rules:
\begin{itemize}\itemsep0em
\item switch the type of each quantifier until
  you reach the last formula without quantifiers;
\item negate the last formula.
\end{itemize}
  One might need to group quantifiers of like type.
\end{rul}

\begin{exc}
  Write the logic statement
   for the negation of Goldbach's conjecture.
\end{exc}
\begin{solution}
  $\exists a\in 2\mathbb{N}^++2$ s.t. $\forall p,q \in \mathbb{P}$, 
   $a\ne p+q$.
\end{solution}

   % \begin{exc}
   %   A weaker version of Goldbach's conjecture states
   %   \emph{Goldbach's conjecture has
   %     at most a finite number of of counterexamples}.
   %   Formulate it into a logical statement
   %    with explicit quatifiers.
   % \end{exc}

   % \begin{exc}
   %   The only even prime is 2.\\
   %   Multiplication of integers is associative. \\
   %   Every positive integer has a unique prime factorization.
   % \end{exc}

\begin{axm}[Contraposition]
  \label{axm:contrapositive}
  A conditional statement is logically equivalent to its
  contrapositive.
  \begin{equation}
    \label{eq:contraposition}
    (\mathsf{A}\Rightarrow \mathsf{B}) \Leftrightarrow
    (\neg \mathsf{B}\Rightarrow \neg \mathsf{A})
  \end{equation}
\end{axm}

\begin{exm}
  \label{exm:contrapositive}
  ``If Jack is a man, then Jack is a human being.''
  is equivalent to ``If Jack is not a human being,
  then Jack is not a man.''
\end{exm}

\begin{exc}
  Draw an Euler diagram of subsets to illustrate Example \ref{exm:contrapositive}.
\end{exc}

%%% Local Variables:
%%% mode: latex
%%% TeX-master: "../notesAlgebraicTopology"
%%% End:
