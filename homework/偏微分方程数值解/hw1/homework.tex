\documentclass[a4paper]{book}
% chktex-file 
% chktex-file 3
\usepackage{geometry}
% make full use of A4 papers
\geometry{margin=1.5cm, vmargin={0pt,1cm}}
\setlength{\topmargin}{-1cm}
\setlength{\paperheight}{29.7cm}
\setlength{\textheight}{25.1cm}

% auto adjust the marginals
\usepackage{marginfix}

\usepackage{amsfonts}
\usepackage{amsmath}
\usepackage{amssymb}
\usepackage{amsthm}
% \usepackage{CJKutf8}   % for Chinese characters
\usepackage{ctex}
\usepackage{enumerate}
\usepackage{graphicx}  % for figures
\usepackage{layout}
\usepackage{multicol}  % multiple columns to reduce number of pages
\usepackage{mathrsfs}
\usepackage{fancyhdr}
\usepackage{subfigure}
\usepackage{tcolorbox}
\usepackage{tikz-cd}
\usepackage{listings}
\usepackage{xcolor} %代码高亮
\usepackage{braket}
\usepackage{graphicx}
\usepackage{subfigure}
% ------------------
% common commands %
% ------------------
% differentiation
\newcommand{\gen}[1]{\left\langle#1 \right\rangle}
\newcommand{\dif}{\mathrm{d}}
\newcommand{\difPx}[1]{\frac{\partial#1}{\partial x}}
\newcommand{\difPy}[1]{\frac{\partial#1}{\partial y}}
\newcommand{\Dim}{\mathrm{D}}
\newcommand{\avg}[1]{\left\langle#1 \right\rangle}
\newcommand{\sgn}{\mathrm{sgn}}
\newcommand{\Span}{\mathrm{span}}
\newcommand{\dom}{\mathrm{dom}}
\newcommand{\Arity}{\mathrm{arity}}
\newcommand{\Int}{\mathrm{Int}}
\newcommand{\Ext}{\mathrm{Ext}}
\newcommand{\Cl}{\mathrm{Cl}}
\newcommand{\Fr}{\mathrm{Fr}}
% group is generated by
\newcommand{\grb}[1]{\left\langle#1 \right\rangle}
% rank
\newcommand{\rank}{\mathrm{rank}}
\newcommand{\Iden}{\mathrm{Id}}

% this environment is for solutions of examples and exercises
\newenvironment{solution}%
{\noindent\textbf{Solution.}}%
{\qedhere}
% the following command is for disabling environments
% so that their contents do not show up in the pdf.
\makeatletter
\newcommand{\voidenvironment}[1]{%
  \expandafter\providecommand\csname env@#1@save@env\endcsname{}%
  \expandafter\providecommand\csname env@#1@process\endcsname{}%
  \@ifundefined{#1}{}{\RenewEnviron{#1}{}}%
}
\makeatother

% ---------------------------------------------
% commands specifically for complex analysis %
% ---------------------------------------------
% complex conjugate
\newcommand{\ccg}[1]{\overline{#1}}
% the imaginary unit
\newcommand{\ii}{\mathbf{i}}
% \newcommand{\ii}{\boldsymbol{i}}
% the real part
\newcommand{\Rez}{\mathrm{Re}\,}
% the imaginary part
\newcommand{\Imz}{\mathrm{Im}\,}
% punctured complex plane
\newcommand{\pcp}{\mathbb{C}^{\bullet}}
% the principle branch of the logarithm
\newcommand{\Log}{\mathrm{Log}}
% the principle value of a nonzero complex number
\newcommand{\Arg}{\mathrm{Arg}}
\newcommand{\Null}{\mathrm{null}}
\newcommand{\Range}{\mathrm{range}}
\newcommand{\Ker}{\mathrm{ker}}
\newcommand{\Iso}{\mathrm{Iso}}
\newcommand{\Aut}{\mathrm{Aut}}
\newcommand{\ord}{\mathrm{ord}}
\newcommand{\Res}{\mathrm{Res}}
% \newcommand{\GL2R}{\mathrm{GL}(2,\mathbb{R})}
\newcommand{\GL}{\mathrm{GL}}
\newcommand{\SL}{\mathrm{SL}}
\newcommand{\Dist}[2]{\left|{#1}-{#2}\right|}

\newcommand\tbbint{{-\mkern-16mu\int}}
\newcommand\tbint{{\mathchar'26 -\mkern-14mu\int}}
\newcommand\dbbint{{-\mkern-20mu\int}}
\newcommand\dbint{{\mathchar'26 -\mkern-26mu\int}}
\newcommand\bint{
  {\mathchoice{\dbint}{\tbint}{\tbint}{\tbint}}
}
\newcommand\bbint{
  {\mathchoice{\dbbint}{\tbbint}{\tbbint}{\tbbint}}
}





% ----------------------------------------
% theorem and theorem-like environments %
% ----------------------------------------
\numberwithin{equation}{chapter}
\theoremstyle{definition}

\newtheorem{thm}{Theorem}[chapter]
\newtheorem{axm}[thm]{Axiom}
\newtheorem{alg}[thm]{Algorithm}
\newtheorem{asm}[thm]{Assumption}
\newtheorem{defn}[thm]{Definition}
\newtheorem{prop}[thm]{Proposition}
\newtheorem{rul}[thm]{Rule}
\newtheorem{coro}[thm]{Corollary}
\newtheorem{lem}[thm]{Lemma}
\newtheorem{exm}{Example}[chapter]
\newtheorem{rem}{Remark}[chapter]
\newtheorem{exc}[exm]{Exercise}
\newtheorem{frm}[thm]{Formula}
\newtheorem{ntn}{Notation}

% for complying with the convention in the textbook
\newtheorem{rmk}[thm]{Remark}


% \lstset{
%	backgroundcolor=\color{red!50!green!50!blue!50},%代码块背景色为浅灰色
%	rulesepcolor= \color{gray}, %代码块边框颜色
%	breaklines=true,  %代码过长则换行
%	numbers=left, %行号在左侧显示
%	numberstyle= \small,%行号字体
%	keywordstyle= \color{blue},%关键字颜色
%	commentstyle=\color{gray}, %注释颜色
%	frame=shadowbox%用方框框住代码块
% }
\lstset{
  columns=fixed,
  numbers=left,                                        % 在左侧显示行号
  numberstyle=\tiny\color{gray},                       % 设定行号格式
  frame=none,                                          % 不显示背景边框
  backgroundcolor=\color[RGB]{245,245,244},            % 设定背景颜色
  keywordstyle=\color[RGB]{40,40,255},                 % 设定关键字颜色
  numberstyle=\footnotesize\color{darkgray},
  commentstyle=\it\color[RGB]{0,96,96},                % 设置代码注释的格式
  stringstyle=\rmfamily\slshape\color[RGB]{128,0,0},   % 设置字符串格式
  showstringspaces=false,                              % 不显示字符串中的空格
  language=c++,                                        % 设置语言
}

% ----------------------
% the end of preamble %
% ----------------------

\begin{document}
\pagestyle{empty}
\pagenumbering{roman}
% 
% \tableofcontents
% \clearpage

% \pagestyle{fancy}
% \fancyhead{}
% \lhead{Qinghai Zhang}
% \chead{Notes on Algebraic Topology}
% \rhead{Fall 2018}


\setcounter{chapter}{0}
\pagenumbering{arabic}
% \setcounter{page}{0}

% --------------------------------------------------------
% uncomment the following to remove these environments
% to generate handouts for students.
% --------------------------------------------------------
% \begingroup
% \voidenvironment{rem}%
% \voidenvironment{proof}%
% \voidenvironment{solution}%


% each chapter is factored into a separate file.

\chapter{Homework 21935004 谭焱}

\section{1.1.1}
\exc Derive estimates for 
 \[ \left| \left( D - \frac{\partial^3}{\partial x^3}\right)
 e^{i\omega x}  \right| \].
 where $ D = D^3_+, D_-D_+^2, D_-^2 D_+, D_0D_+D_-$.

 \begin{solution}
    By 
\begin{align*}
    &E^3 e^{i\omega x} = (1 + 3i\omega h - \frac{9}{2}\omega^2h^2 - 
    \frac{27}{6}i \omega^3h^3 + \frac{81}{24} \omega^4h^4 +
     \mathcal{O}(\omega^4 h^4)) e^{i\omega x} \\
    &E^2 e^{i\omega x} = (1 + 2i\omega h - \frac{4}{2}\omega^2h^2 - 
    \frac{8}{6}i \omega^3h^3 + \frac{16}{24} \omega^4h^4 +
     \mathcal{O}(\omega^4 h^4))  e^{i\omega x} \\
    &E e^{i\omega x} = (1 + i\omega h - \frac{1}{2}i\omega2h^2 - 
    \frac{1}{6}i \omega^3h^3 + \frac{1}{24} \omega^4h^4 
     \mathcal{O}(\omega^4 h^4))  e^{i\omega x} \\
    &E^{-1} e^{i\omega x} = (1 - i\omega h - \frac{1}{2}i\omega2h^2 + 
     \frac{1}{6}i \omega^3h^3 + \frac{1}{24} \omega^4h^4 
      \mathcal{O}(\omega^4 h^4))  e^{i\omega x} \\
    &E^{-2} e^{i\omega x} = (1 - 2i\omega h - \frac{4}{2}i\omega2h^2 + 
     \frac{8}{6}i \omega^3h^3 + \frac{16}{24} \omega^4h^4 
      \mathcal{O}(\omega^4 h^4))  e^{i\omega x} \\
    &E^3 e^{i\omega x} = (1 - 3i\omega h - \frac{9}{2}\omega^2h^2 + 
      \frac{27}{6}i \omega^3h^3 + \frac{81}{24} \omega^4h^4 +
       \mathcal{O}(\omega^4 h^4)) e^{i\omega x} \\
\end{align*}
\begin{enumerate}
    \item $D = D_+^3$ 
        \begin{align*}
            \left| \left( D_+^3 - \frac{\partial^3}{\partial x^3}\right)
            e^{i\omega x}  \right| &= 
            \left| \left( E^3 - 3E^2 + 3E - E_0 + i (\omega h)^3 \right)
            e^{i\omega x} / h^3  \right| \\
            &=\left| (\frac{3}{2} \omega^4 h^4 +
             \mathcal{O} (\omega^4 h^4) )  e^{i\omega x} \right| 
             =  \mathcal{O} (\omega^3 h^3)
        \end{align*}

        \item $D = D_- D_+^2$ 
        \begin{align*}
            \left| \left(D_- D_+^2 - \frac{\partial^3}{\partial x^3}\right)
            e^{i\omega x}  \right| &= 
            \left| \left( E^2 - 3E + 3E_0 - E^{-1} + i (\omega h)^3 \right)
            e^{i\omega x} / h^3  \right| \\
            &=\left| (\frac{1}{2} \omega^4 h^4 +
             \mathcal{O} (\omega^4 h^4) )  e^{i\omega x} \right|
             =  \mathcal{O} (\omega^3 h^3)
        \end{align*}
        
        \item $D = D_-^2 D_+$ 
        \begin{align*}
            \left| \left(D_-^2 D_+ - \frac{\partial^3}{\partial x^3}\right)
            e^{i\omega x}  \right| &= 
            \left| \left( E - 3E_0 + 3E^{-1} - E^{-2} + i (\omega h)^3 \right)
            e^{i\omega x} / h^3  \right| \\
            &=\left| (-\frac{1}{2} \omega^4 h^4 +
             \mathcal{O} (\omega^4 h^4) )  e^{i\omega x} \right|
             =  \mathcal{O} (\omega^3 h^3)
        \end{align*}

        \item $D = D_0 D_- D_+$ 
        \begin{align*}
            \left| \left(D_0 D_- D_+ - \frac{\partial^3}{\partial x^3}\right)
            e^{i\omega x}  \right| &= 
            \left| \left( E^2 - 2E + 2E^{-1} + E^{-2} + i (\omega h)^3 \right)
            e^{i\omega x} / h^3  \right| \\
            &=\left| (
            \mathcal{O} (\omega^4 h^4) )  e^{i\omega x} \right|
            =  \mathcal{O} (\omega^4 h^4)
        \end{align*}
\end{enumerate}

 \end{solution}

\section{1.3.1}
\exc Compute $\lVert D_+ D_- \rVert_h$.

\begin{solution}
    Firstly, we have 
    \begin{align*}
        \lVert D_+ D_- \rVert_h \leq
         \lVert D_+ \rVert_h \lVert D_- \rVert_h
         = \frac{4}{h^2}
    \end{align*}
    Then, let $u_j = (-1)^j$, consider
    \begin{align*}
        \lVert D_+ D_- u \rVert_h^2 
        &= \sum_j ((E - 2E_0 + E^{-1})u_j)^2 / h^4 \\
        &= \sum_j 16/ h^4 = \sum_j 16/ h^4 \lVert u \rVert_h^2
    \end{align*}.
    That means 
    \[\lVert D_+ D_- \rVert_h \geq
     \sqrt{\frac{\lVert D_+ D_- u \rVert_h^2 }
     { \lVert u \rVert_h^2}} = \frac{2}{h^2} \].
    In summary, $\lVert D_+ D_- \rVert_h = \frac{2}{h^2}$. 
\end{solution}

\end{document}