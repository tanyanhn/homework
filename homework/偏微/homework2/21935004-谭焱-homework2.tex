\documentclass[a4paper]{book}

\usepackage{geometry}
% make full use of A4 papers
\geometry{margin=1.5cm, vmargin={0pt,1cm}}
\setlength{\topmargin}{-1cm}
\setlength{\paperheight}{29.7cm}
\setlength{\textheight}{25.1cm}

% auto adjust the marginals
\usepackage{marginfix}

\usepackage{amsfonts}
\usepackage{amsmath}
\usepackage{amssymb}
\usepackage{amsthm}
%\usepackage{CJKutf8}   % for Chinese characters
\usepackage{ctex}
\usepackage{enumerate}
\usepackage{graphicx}  % for figures
\usepackage{layout}
\usepackage{multicol}  % multiple columns to reduce number of pages
\usepackage{mathrsfs}  
\usepackage{fancyhdr}
\usepackage{subfigure}
\usepackage{tcolorbox}
\usepackage{tikz-cd}
\usepackage{listings}
\usepackage{xcolor} %代码高亮
%------------------
% common commands %
%------------------
% differentiation
\newcommand{\gen}[1]{\left\langle #1 \right\rangle}
\newcommand{\dif}{\mathrm{d}}
\newcommand{\difPx}[1]{\frac{\partial #1}{\partial x}}
\newcommand{\difPy}[1]{\frac{\partial #1}{\partial y}}
\newcommand{\Dim}{\mathrm{D}}
\newcommand{\avg}[1]{\left\langle #1 \right\rangle}
\newcommand{\sgn}{\mathrm{sgn}}
\newcommand{\Span}{\mathrm{span}}
\newcommand{\dom}{\mathrm{dom}}
\newcommand{\Arity}{\mathrm{arity}}
\newcommand{\Int}{\mathrm{Int}}
\newcommand{\Ext}{\mathrm{Ext}}
\newcommand{\Cl}{\mathrm{Cl}}
\newcommand{\Fr}{\mathrm{Fr}}
% group is generated by
\newcommand{\grb}[1]{\left\langle #1 \right\rangle}
% rank
\newcommand{\rank}{\mathrm{rank}}
\newcommand{\Iden}{\mathrm{Id}}

% this environment is for solutions of examples and exercises
\newenvironment{solution}%
{\noindent\textbf{Solution.}}%
{\qedhere}
% the following command is for disabling environments
%  so that their contents do not show up in the pdf.
\makeatletter
\newcommand{\voidenvironment}[1]{%
  \expandafter\providecommand\csname env@#1@save@env\endcsname{}%
  \expandafter\providecommand\csname env@#1@process\endcsname{}%
  \@ifundefined{#1}{}{\RenewEnviron{#1}{}}%
}
\makeatother

%---------------------------------------------
% commands specifically for complex analysis %
%---------------------------------------------
% complex conjugate
\newcommand{\ccg}[1]{\overline{#1}}
% the imaginary unit
\newcommand{\ii}{\mathbf{i}}
%\newcommand{\ii}{\boldsymbol{i}}
% the real part
\newcommand{\Rez}{\mathrm{Re}\,}
% the imaginary part
\newcommand{\Imz}{\mathrm{Im}\,}
% punctured complex plane
\newcommand{\pcp}{\mathbb{C}^{\bullet}}
% the principle branch of the logarithm
\newcommand{\Log}{\mathrm{Log}}
% the principle value of a nonzero complex number
\newcommand{\Arg}{\mathrm{Arg}}
\newcommand{\Null}{\mathrm{null}}
\newcommand{\Range}{\mathrm{range}}
\newcommand{\Ker}{\mathrm{ker}}
\newcommand{\Iso}{\mathrm{Iso}}
\newcommand{\Aut}{\mathrm{Aut}}
\newcommand{\ord}{\mathrm{ord}}
\newcommand{\Res}{\mathrm{Res}}
%\newcommand{\GL2R}{\mathrm{GL}(2,\mathbb{R})}
\newcommand{\GL}{\mathrm{GL}}
\newcommand{\SL}{\mathrm{SL}}
\newcommand{\Dist}[2]{\left|{#1}-{#2}\right|}

\newcommand\tbbint{{-\mkern -16mu\int}}
\newcommand\tbint{{\mathchar '26\mkern -14mu\int}}
\newcommand\dbbint{{-\mkern -19mu\int}}
\newcommand\dbint{{\mathchar '26\mkern -18mu\int}}
\newcommand\bint{
	{\mathchoice{\dbint}{\tbint}{\tbint}{\tbint}}
}
\newcommand\bbint{
	{\mathchoice{\dbbint}{\tbbint}{\tbbint}{\tbbint}}
}





%----------------------------------------
% theorem and theorem-like environments %
%----------------------------------------
\numberwithin{equation}{chapter}
\theoremstyle{definition}

\newtheorem{thm}{Theorem}[chapter]
\newtheorem{axm}[thm]{Axiom}
\newtheorem{alg}[thm]{Algorithm}
\newtheorem{asm}[thm]{Assumption}
\newtheorem{defn}[thm]{Definition}
\newtheorem{prop}[thm]{Proposition}
\newtheorem{rul}[thm]{Rule}
\newtheorem{coro}[thm]{Corollary}
\newtheorem{lem}[thm]{Lemma}
\newtheorem{exm}{Example}[chapter]
\newtheorem{rem}{Remark}[chapter]
\newtheorem{exc}[exm]{Exercise}
\newtheorem{frm}[thm]{Formula}
\newtheorem{ntn}{Notation}

% for complying with the convention in the textbook
\newtheorem{rmk}[thm]{Remark}


%\lstset{
%	backgroundcolor=\color{red!50!green!50!blue!50},%代码块背景色为浅灰色
%	rulesepcolor= \color{gray}, %代码块边框颜色
%	breaklines=true,  %代码过长则换行
%	numbers=left, %行号在左侧显示
%	numberstyle= \small,%行号字体
%	keywordstyle= \color{blue},%关键字颜色
%	commentstyle=\color{gray}, %注释颜色
%	frame=shadowbox%用方框框住代码块
%}
\lstset{
	columns=fixed,       
	numbers=left,                                        % 在左侧显示行号
	numberstyle=\tiny\color{gray},                       % 设定行号格式
	frame=none,                                          % 不显示背景边框
	backgroundcolor=\color[RGB]{245,245,244},            % 设定背景颜色
	keywordstyle=\color[RGB]{40,40,255},                 % 设定关键字颜色
	numberstyle=\footnotesize\color{darkgray},           
	commentstyle=\it\color[RGB]{0,96,96},                % 设置代码注释的格式
	stringstyle=\rmfamily\slshape\color[RGB]{128,0,0},   % 设置字符串格式
	showstringspaces=false,                              % 不显示字符串中的空格
	language=c++,                                        % 设置语言
}

%----------------------
% the end of preamble %
%----------------------

\begin{document}
\pagestyle{empty}
\pagenumbering{roman}

%\tableofcontents
\clearpage



\setcounter{chapter}{0}
\pagenumbering{arabic}
% \setcounter{page}{0}

% --------------------------------------------------------
% uncomment the following to remove these environments 
%  to generate handouts for students.
% --------------------------------------------------------
% \begingroup
% \voidenvironment{rem}%
% \voidenvironment{proof}%
% \voidenvironment{solution}%


% each chapter is factored into a separate file.

\chapter{Homework2 21935004 谭焱}
%\begin{lstlisting}
%int main(){
%  double d;
%  int i;
%  return i;
%}
%\end{lstlisting}
% The main ingredients of snacks are sugar and fat;
%  the main ingredients of math are logic and set theory.


\begin{multicols}{2}
\setlength{\columnseprule}{0.2pt}  

\section{chapter 2.5}
\paragraph{5}
We say $v \in C^2(\bar{U})$ is \textit{subharmonic} if 
\begin{equation} \label{2:5:1}
	-\Delta v \leq 0 \qquad \text{in} \; U.
\end{equation}
\begin{itemize}
	\item[(a)] Prove for subharmonic $v$ that 
	\begin{equation} \label{2:5:2} 
	v(x) \leq \dbbint_{B(x,r)} v dy \qquad \text{for all } B(x,t) \subset U.
	\end{equation}
	\item[(b)] Prove that there for $ max_{\bar{U}} v = max_{\partial\bar{U}} v. $
	\item[(c)] Let $\phi : \mathbb{R} \rightarrow \mathbb{R} $ be smooth and convex. Assume u is harmonic and $ v := \phi(u)$. Prove $v$ is subharmonic.
	\item[(d)] Prove $v := |Du|^2$ is subharmonic, whenever $u$ is harmonic.
\end{itemize}

\begin{solution}
	\begin{itemize}
		\item[(a)] Already have two equation below
		$$ \dbbint_{B(x,r)} v dy = \int_{0}^{r}(\int_{\partial B(x,r)} u dS) / \alpha (n) x^n dx = \dbbint_{\partial B(x,r)} v dS $$
		$$ \dbbint_{\partial B(x,r)} v dS = \dbbint_{\partial B(0,1)}Du(x + rz) \cdot z dS(z) = \phi (r) $$
		and differential $\phi(r)$ get 
		\begin{equation} 
		\begin{aligned} 
		 \phi \prime (r) &= \dbbint_{\partial B(x,r)} Du(y) \cdot \frac{y -x}{r} dS(y) \\
		&= \dbbint_{B(x,r)} \frac{\partial u}{\partial \nu} dS(y) \\
		&= \frac{r}{n} \dbbint_{B(x,r)} \Delta v(y) dy \\
		&\geq 0
		\end{aligned} 
		\end{equation} 
		so $\phi(r) \geq \phi(0) $ indicate that \ref{2:5:1}
		
		\item[(b)] Because \ref{2:5:1}.we have $v(x)$ always small than a value in $B(x,r) \in U$,
		
		do $max_{\bar{U}} \in \partial U$ means $max_{\bar{U}} = max_{\partial U} v. $
		
		\item[(c)] Differential v get
		$$ \partial_i v = \partial \phi(u) \cdot \partial_i u $$
		
		then 
		\begin{equation}
			\begin{aligned}
			\sum_{i = 1}^{n} \partial_i \partial_i v &=\sum_{i = 1}^{n} \partial^2 \phi(u) \cdot (\partial_i u)^2 + \partial \phi(u) \cdot (\partial_i)^2 u \\ 
			&=\sum_{i = 1}^{n} \partial^2 \phi(u) \cdot (\partial_i u)^2 \\
			&\geq 0 ,
			\end{aligned}
		\end{equation}
		which means $v$ is subharmonic.
		
		\item[(d)]The same as (c) have
		$$ \partial_i v = \partial_i(\sum_{j = 1}^{n} u_j^2) = \sum_{j = 1}^{n} 2u_j \cdot u_{ji} $$
		
		then 
		\begin{equation}
			\begin{aligned}
				\sum_{i = 1}^{n} \partial_i \partial_i v &= \sum_{i = 1}^n \sum_{j = 1}^n(2(u_{ij})^2 + 2u_ju_{jii}) \\
				&\geq \sum_{i = 1}^n \sum_{j = 1}^n 2u_j u_{jii}
			\end{aligned}
		\end{equation}
		$u$ is harmonic conclude $ \sum_{i = 1}^{n} u_{ii} = 0 $ ,differential j have $ \sum_{i = 1}^{n} u_{iij} = 0 $. Combining with above. $\Delta v \geq 0$.
	\end{itemize}
\end{solution}


\paragraph{6}
Let $U$ be a bounded, open subset of $\mathbb{R}^n$. Prove that there exists a constant $C$ , depending only on $U$ ,such that 
\begin{equation}\label{2:6:1}
	max_{\bar{U}} |u| \leq C(max_{\partial U} |g| + max_{\bar{U}} |f|)
\end{equation}
whenever $u$ is a smooth solution of 
\begin{equation}\label{2:6:2}
	\begin{cases}
	\begin{aligned}
	-\Delta u &= f \qquad \text{in } U\\
	u &= g \qquad \text{on } \partial U.
	\end{aligned}
	\end{cases}
\end{equation}

Because $\Delta (\frac{|x|^2}{2n}\lambda) = \lambda $ and $U$ is bounded .Let $C > \frac{|x|^2}{xn} \text{ for }x \in U$ and $\lambda = max_{\bar{U}} |f|$ .So set $\phi = u + \frac{|x|^2}{2n} \lambda$ and 
\begin{equation}
	\Delta \phi = \Delta (u + \frac{|x|^2}{2n} \lambda) = f + \lambda \geq 0
\end{equation}

By problem 4 , $\phi$ is subharmonic, and accord to \ref{2:5:2}
\begin{equation}
	\phi \leq w(n)r^n max_{\partial U} |g|.
\end{equation}
$r$ is the biggest $r$ for $ B(x.r) \in U$ . Let $C > w(n)r^n \quad \forall x \in U$ get \ref{2:6:1}.


\paragraph{10 (Reflection principle)}
\begin{itemize}
	\item[(a)] Let $U^+$ denote the open half-ball $ \{x \in \mathbb{R}^n | \; |x| < 1,\; x_n > x\} $. Assume $u \in C^2(\overline{U^+})$ is harmonic in $U^+$ , with $u = 0$ on $ \partial U^+ \cap {x_n = 0}$. Set
	\begin{equation}\label{2:10:1}
	v(x) := 
		\begin{cases}
			\begin{aligned}
				&u(x)         &\text{if } x_n \geq 0\\
				&-u(x_1, ... ,x_{n-1}, -x_n)  &\text{if } x_n \le 0
			\end{aligned}
		\end{cases}
	\end{equation}
	for $x \in U = B^0(0,1)$. Prove $v \in C^2(U)$ and thus $v$ is harmonic within $U$ .
	
	\item[(b)]Now assume only that $u \in C^2(U^+)\cap C(\overline{U^+})$. Show that $v$ is harmonic within $U$. 
\end{itemize}

\begin{solution}
\begin{itemize}
	\item[(a)]$u$ is harmonic in $C^2(U^+)$.calculate v 
	\begin{equation}
		\Delta v =
		\begin{cases}
		\begin{aligned}
		&\Delta u(x)  &\text{if } x_n > 0 \\
		&\Delta -u(x)  &\text{if } x_n < 0
		\end{aligned}
		\end{cases}
		= 0
	\end{equation}
	because $u \in C^2(\overline{U^+})$, $\Delta u$ is continue in $x \text{ satisfy } x_n = 0$. $\Delta v = 0  \text{ if } x_n = 0$.
	
	Finally, $\Delta v = 0 \; \forall x \in U$ .Means $v$ is harmonic within $U$.
	
	\item[(b)] Using Poisson's formula for boundary $\partial U$. get solution
	$$ f(x) = \frac{1 - |x|^2}{n\alpha (n)}\int_{\partial B(x,1)} \frac{u(y)}{|x-y|^n} dS(y) $$
	
	It is easy to confirm that $f(x) = 0 = u(x) \quad \forall x \text{ satisfy }x_n = 0$,because of $u(y) = -u(-y)$.
	
	$f(x)$ and $u(x)$ is harmonic in $U^+$, conclude $ f(x) - u(x) $ is harmonic.linking $f(x) - u(x) = 0 \text{ in } \partial U^+$, so according to strong maximum principle. $f(x) - u(x) = 0 \; \forall x \in \overline{U^+}$.
	
	The same to get $f(x) + u(x) = 0 \text{ in } \partial U^+$.In summary. $f(x) - v(x) = 0 \; \forall x \in \bar{U}$, which means $v(x)$ is harmonic within $U$.
\end{itemize}
\end{solution}


\paragraph{11 (Kelvin transform for Laplace's equation)}
The \textit{Kelvin transform} $\mathcal{K}u = \bar{u}$ of a function $u : \mathbb{R}^n \rightarrow \mathbb{R}$ is 
\begin{equation}\label{2:11:1}
	\bar{u}(x) := u(\bar{x})|\bar{x}|^{n-2} = u(x/|x|)|x|^{2-n}  \qquad (x \neq 0),
\end{equation}
where $\bar{x} = x/|x|^2$. Show that if $u$ is harmonic, then so is $\bar{u}$.

\begin{solution}
	Calculation differential and laplace of $\bar{x} = \frac{(x_1, x_2,...,x_n)}{\sum_{i = 1}^{n} x_i^2} $ 
	\begin{equation}\label{2:11:2}
		D_{x_i} \bar{x}_j = (\delta_{ij} \cdot |x|^2 - 2x_ix_j)/ |x|^4 
	\end{equation}
	where $\delta_{ij} = 
	\begin{cases}
	1 \qquad \text{if } i = j \\
	0 \qquad \text{if } i \neq j,
	\end{cases}$
	so $D\bar{x}$ is a matrix $A = |x|^2 * I - x^T \cdot x$.
	By product $D_x \bar{x}(D_x \bar{x})^T$ is a matrix $B = AA^T$ satisfy
	\begin{equation}\label{2:11:3}
	\begin{aligned} 
		B_{ij}|x|^8 &= \sum_{k = 1}^{n}(\delta_{ik} \cdot |x|^2 - 2x_{i} x_k)(\delta_{jk} \cdot |x|^2 - 2x_{j} x_k)\\
		&=
		\begin{cases}
		|x|^4 \qquad &\text{if } i = j \\ 
		0 \qquad &\text{if } i \neq j.
		\end{cases}
	\end{aligned} 
	\end{equation}
	Now calculate $\Delta \bar{u}(x)$
	\begin{equation}\label{2:11:4}
		\begin{aligned}
		\Delta \bar{u}(x) =& \sum_{i = 1}^{n} \partial_i \partial_i(u(\bar{x})|x|^{2-n}) \\
		=&  \sum_{i = 1}^{n} \partial_i (\sum_{j = 1}^{n}\partial_j (u(\bar{x}))\partial_i(\bar{x}_j)|x|^{2-n} + u(\bar{x})\partial_i(|x|^{2-n}))  \\
		=& \sum_{i = 1}^{n}((\partial_i)^2(|x|^{2-m})u(\bar{x})|x|^{1-n} + \\ &2\partial_i(|x|^{2-n})\sum_{j = 1}^{n}\partial_j (u(\bar{x}))\partial_i(\bar{x}_j) + \\
		&\sum_{j = 1}^{n}\partial_j (u(\bar{x})) \partial_i^2(\bar{x}_j)|x|^{2-n} +\\
		&\sum_{j = 1}^{n}\sum_{k = 1}^{n}\partial_k\partial_j(u(\bar{x})) \partial_i(\bar{x}_k)\partial_i(\bar{x}_j)|x|^{2-n}   ). 
		\end{aligned}
	\end{equation}
	According to $|x|^{2-n}$ is harmonic,the first of \ref{2:11:4} is 0.
	According to \ref{2:11:3} and \ref{2:11:2} ,the fourth of $\ref{2:11:4} = \sum_{i = 1}^{n} tr{\partial(\bar{x}) \partial^2u \partial(\bar{x})}|x|^{2-n} = \sum_{i = 1}^{n} \partial_i^2 u |x|^{-2-n} = 0$. 
	the second of \ref{2:11:4}
	\begin{equation}
	\begin{aligned} 
		&\sum_{i = 1}^{n} 2\partial_i(|x|^{2-n})\sum_{j = 1}^{n}\partial_j (u(\bar{x}))\partial_i(\bar{x}_j) \\
		&= 2 Du D(\bar{x}) D(|x|^{2-n}) \\
		&= 2 Du |x|^{-2}(I - \frac{xx^T}{|x|^2}) \cdot (2-n)|x|^{1-n}\frac{x^t}{|x|} \\
		&= 2(2-n)|x|^{-2-n}(x - 2x) \cdot Du \\
		&= -2(2-n)|x|^{-2-n} x \cdot Du,
	\end{aligned} 
	\end{equation}
	the third of \ref{2:11:4} according to \ref{2:11:2} 
	\begin{equation}
		\begin{aligned}
		&\partial_i^2(\bar{x}_j) \\
		&=\begin{cases}
		\frac{(\delta_{ij} (2) x_i |x|^4 - 2x_j |x|^4 - \delta_{ij} (4) x_i |x|^4) + (8)x_i^2 x_j |x|^2}{|x|^8} \qquad \text{if } i \neq j \\
		\frac{(\delta_{ij} (2) x_i |x|^4 - 4x_i |x|^4 - \delta_{ij} (4) x_i |x|^4) + (8)x_i^2 x_j |x|^2}{|x|^8} \qquad \text{if } i = j 
		\end{cases} 
		\end{aligned}
	\end{equation}
	so 
	\begin{equation}
		\sum_{j = 1}^{n}\partial_j (u(\bar{x})) \partial_i^2(\bar{x}_j)|x|^{2-n}
		= Du 2(2-n) |x|^{-2-n} \cdot x.
	\end{equation}
	In summary, $\Delta \bar{u}(x) = 0 \text{,when} \Delta u(x) = 0$. 
\end{solution}
%\label{sec:logic}
%\begin{defn}
  A \emph{set} ${\cal S}$
  is a collection of \emph{distinct} objects $x$'s,
   often denoted with the following notation
   \begin{equation}
     \label{eq:setNotation}
     {\cal S} = \{ x\ |\ \text{ the conditions that $x$ satisfies. } \}.
   \end{equation}
\end{defn}

\begin{ntn}
$\mathbb{R}, \mathbb{Z}, \mathbb{N}, \mathbb{Q}, \mathbb{C}$
 denote 
 the sets of real numbers, integers, natural numbers,
 rational numbers and complex numbers, respectively.
$\mathbb{R}^+, \mathbb{Z}^+, \mathbb{N}^+, \mathbb{Q}^+$
 the sets of positive such numbers.
\end{ntn}

 \begin{defn}
   ${\cal S}$ is a \emph{subset} of ${\cal U}$,
    written as ${\cal S}\subseteq {\cal U}$,
   if and only if (iff) $x\in {\cal S}$ $\Rightarrow$ $x\in {\cal U}$.
   ${\cal S}$ is a \emph{proper subset} of ${\cal U}$,
    written as ${\cal S}\subset {\cal U}$,
    if ${\cal S}\subseteq {\cal U}$
    and $\exists x\in {\cal U}$ s.t. $x\not\in{\cal S}$.
 \end{defn}

\begin{defn}[Statements of first-order logic]
\label{def:uni_exist}
A \emph{universal statement} is a logic statement 
 of the form
\begin{equation}
  \mathsf{U} = (\forall x\in {\cal S},\ \mathsf{A}(x) ).
\end{equation}
An \emph{existential statement} has the form
\begin{equation}
  \mathsf{E} = (\exists x\in {\cal S},\text{ s.t. } \mathsf{A}(x)),
\end{equation}
 where 
 $\forall$ (``for each'') and $\exists$ (``there exists'')
 are the \emph{quantifiers}, ${\cal S}$ is a set,
 ``s.t.'' means ``such that,''
 and $\mathsf{A}(x)$ is the \emph{formula}.\\
A statement of \emph{implication/conditional}
 has the form
 \begin{equation}
   \mathsf{A}\Rightarrow \mathsf{B}.
 \end{equation}
\end{defn}

 \begin{exm}
   Universal and existential statements:\\
   $\forall x\in[2,+\infty)$, $x>1$;\\
   $\forall x\in \mathbb{R}^+$, $x>1$;\\
   $\exists p,q\in \mathbb{Z}, \text{ s.t. } p/q = \sqrt{2}$;\\
   $\exists p,q\in \mathbb{Z}, \text{ s.t. } \sqrt{p} = \sqrt{q}+1$.
 \end{exm}

\begin{defn}
  \emph{Uniqueness quantification}
   or \emph{unique existential quantification},
   written $\exists!$ or $\exists_{=1}$, 
   indicates that exactly one object with a certain property exists.
\end{defn}

\begin{exc}
  Express the logic statement $\exists! x, \text{ s.t. } \mathsf{A}(x)$
   with $\exists$, $\forall$, and $\Leftrightarrow$.
\end{exc}
\begin{solution}
  $\exists x \text{ s.t. }\forall y, \mathsf{A}(y) \Leftrightarrow x=y.$
\end{solution}

 \begin{rem}
A logic statement is either true or false.
There is no such thing that
 a logic statement is sometimes true and sometimes false.
To prove a universal statement,
 conceptually we have to verify the statement
 for all elements in the set.
To deny a universal statement,
 we only need to show a counterexample.
To prove an existential statement,
 we only need to show an instance.
To deny an existential statement,
 conceptually we have to show that the statement holds
 for none of the elements.
 \end{rem}

 \begin{rem}
   In Definition \ref{def:uni_exist},
    the formula $\mathsf{A}(x)$ itself
    might also be a logic statement.
   Hence universal and existential statements
    might be nested.
   This observation leads to the next definition.
 \end{rem}

 \begin{defn}
   A \emph{universal-existential statement} is a logic statement 
   of the form
   \begin{equation}
     \mathsf{U}_E =
     (\forall x\in {\cal S},\ \exists y\in {\cal T}
     \text{ s.t. } \mathsf{A}(x,y)).
   \end{equation}
   An \emph{existential-universal statement} has the form
   \begin{equation}
     \mathsf{E}_U =
     (\exists y\in {\cal T},\text{ s.t. } \forall x\in {\cal S},\ 
     \mathsf{A}(x,y)).
   \end{equation}
 \end{defn}

 \begin{exm}
   True or false:\\
   $\forall x\in[2,+\infty)$, $\exists y\in \mathbb{Z}^+$ s.t. $x^y<10^5$;\\
   $\exists y\in \mathbb{R}$ s.t.
   $\forall x\in[2,+\infty)$, $x>y$;\\
   $\exists y\in \mathbb{R}$ s.t.
   $\forall x\in[2,+\infty)$, $x<y$.
 \end{exm}

\begin{exc}
  [Translating an English statement into a logic statement]
  Goldbach's conjecture states
   \emph{every even natural number greater than 2
     is the sum of two primes}.
 \end{exc}
\begin{solution}
  Let $\mathbb{P}\subset \mathbb{N}^+$
   denote the set of prime numbers.
  Then Goldbach's conjecture is
  \begin{displaymath}
  \forall a\in 2\mathbb{N}^++2,
   \exists p,q\in \mathbb{P} \text{ s.t. } a=p+q.\qedhere
  \end{displaymath}
\end{solution}

\begin{thm}
   The existential-universal statement
    implies the corresponding universal-existential statement,
    but not vice versa.
 \end{thm}

 \begin{exm}[Translating a logic statement to an English statement]
   Let ${\cal S}$ be the set of all human beings.\\
   $U_E=$($\forall p\in{\cal S}, \exists q\in{\cal S}$ s.t. $q$ is $p$'s mom.)
   \\
   $E_U$=( $\exists q\in{\cal S}$ s.t.
   $\forall p\in{\cal S}, $ $q$ is $p$'s mom.)\\
   $U_E$ is probably true, but $E_U$ is certainly false. \\
   If $E_U$ were true, then $U_E$ would be true. why?
 \end{exm}

\begin{axm}[First-order negation of logical statements]
  The negations of the statements in Definition \ref{def:uni_exist}
  are
  \begin{align}
  \neg \mathsf{U} &= (\exists x\in {\cal S},\text{ s.t. }
  \neg \mathsf{A}(x)).
  \\
  \neg \mathsf{E} &= (\forall x\in {\cal S},\ 
  \neg \mathsf{A}(x)).
  \end{align}
\end{axm}

\begin{rul}
  The negation of a more complicated logic statement
   abides by the following rules:
\begin{itemize}\itemsep0em
\item switch the type of each quantifier until
  you reach the last formula without quantifiers;
\item negate the last formula.
\end{itemize}
  One might need to group quantifiers of like type.
\end{rul}

\begin{exc}
  Write the logic statement
   for the negation of Goldbach's conjecture.
\end{exc}
\begin{solution}
  $\exists a\in 2\mathbb{N}^++2$ s.t. $\forall p,q \in \mathbb{P}$, 
   $a\ne p+q$.
\end{solution}

   % \begin{exc}
   %   A weaker version of Goldbach's conjecture states
   %   \emph{Goldbach's conjecture has
   %     at most a finite number of of counterexamples}.
   %   Formulate it into a logical statement
   %    with explicit quatifiers.
   % \end{exc}

   % \begin{exc}
   %   The only even prime is 2.\\
   %   Multiplication of integers is associative. \\
   %   Every positive integer has a unique prime factorization.
   % \end{exc}

\begin{axm}[Contraposition]
  \label{axm:contrapositive}
  A conditional statement is logically equivalent to its
  contrapositive.
  \begin{equation}
    \label{eq:contraposition}
    (\mathsf{A}\Rightarrow \mathsf{B}) \Leftrightarrow
    (\neg \mathsf{B}\Rightarrow \neg \mathsf{A})
  \end{equation}
\end{axm}

\begin{exm}
  \label{exm:contrapositive}
  ``If Jack is a man, then Jack is a human being.''
  is equivalent to ``If Jack is not a human being,
  then Jack is not a man.''
\end{exm}

\begin{exc}
  Draw an Euler diagram of subsets to illustrate Example \ref{exm:contrapositive}.
\end{exc}

%%% Local Variables:
%%% mode: latex
%%% TeX-master: "../notesAlgebraicTopology"
%%% End:




\end{multicols}


\end{document}


%%% Local Variables: 
%%% mode: latex
%%% TeX-master: t
%%% End: 

% LocalWords:  FPN underflows denormalized FPNs matlab eps IEEE iff
% LocalWords:  cardinality significand quadratically bijection unary
%  LocalWords:  contractive bijective postcondition invertible arity
%  LocalWords:  subspaces surjective injective monomials additivity
%  LocalWords:  nullary Abelian abelian finitary eigenvectors adjoint
%  LocalWords:  eigenvector nullspace Hermitian unitarily multiset
%  LocalWords:  nonsingular nonconstant homomorphism homomorphisms
%  LocalWords:  isomorphically indeterminates subfield isomorphism
%  LocalWords:  nondefective diagonalizable contrapositive cofactor
%  LocalWords:  submatrix nilpotent positivity orthonormal extremum
%  LocalWords:  Jacobian nonsquare semidefinite nonnegative RHS LLS
%  LocalWords:  roundoff closedness
